%
% ================================ 
% | LOCAL SETTINGS               |
% ================================

% >> Überschriften
\renewcommand{\aufgabe}[1]{{\Large \textsf{\textbf{Aufgabe #1}}}}
\renewcommand{\teilaufgabe}[1]{\textsf{\textit{{\large Teilaufgabe (#1)}}}}
% Überschriften bereits in tex_theoreme definiert, hier überschrieben

% >> Sichtbarkeit von Countern
\numberwithin{themcount}{section}
\counterwithout{section}{chapter}

% >> Hervorhebungen
\newcommand{\begriff}[1]{\textit{#1}}

% >> Nummerierung der Gleichungen
\renewcommand*{\theequation}{\arabic{equation}}

% >> Aufzählungszeichen und Nummerierung
\renewcommand{\labelitemi}{$\vartriangleright$}
\renewcommand{\labelenumi}{\alph{enumi})}

% >> Variantionen des Dreiecks als Aufzählungszeichen
% $\blacktriangleright$
% $\vartriangleright$
% $\triangleright$ 

% >> Aufzählungszeichen für Formel in der gleichen Zeile
\def\Item{\item~\vspace{-1\normalbaselineskip}}

% >> Fußnoten
\renewcommand{\thefootnote}{\alph{footnote}}

% >> Multiplkationszeichen: Malpunkt statt *
\DeclareMathSymbol{*}{\mathbin}{symbols}{"01}

% >> mathemtische Zeichen
\renewcommand{\follows}{\enspace \Rightarrow \enspace}
\renewcommand{\equivalent}{\enspace \Leftrightarrow \enspace}

