\section{Erinnerung und Beispiele}

\begin{erinnerung}
	Eine Gruppe ist ein Paar $(G,\star)$ bestehend aus einer Menge $G$ und einer Abbildung $\star : G \times G \to G$, das die Axiome Assoziativität, Existenz eines neutralen Elements und Existenz eines inversen Elements erfüllt. Wir schreiben auch $G$ für $(G,\star)$. Die Gruppe ist abelsch, wenn $g \star h = h \star g$ für alle $g,h \in G$ gilt. Eine allgemeine Gruppe schreiben wir multiplikativ mit neutralem Element $1$, abelsche Gruppen auch additiv mit neutralem Element $0$. \\
	Eine Teilmenge $H \subseteq G$ ist eine Untergruppe von $G$, in Zeichen $H \leq G$, wenn $H \neq \emptyset$ und $H$ abgeschlossen ist unter der Verknüpfung und dem Bilden von Inversen. \\
	Wir schreiben $1$ (bzw. $0$) für die triviale Untergruppe $\{1\}$ (bzw. $\{0\}$) von $G$. \\
	Eine Abbildung $\phi : G \to G'$ zwischen Gruppen ist ein Gruppenhomomorphismus, wenn
	\begin{align*}
		\phi(g_1 * g_2) = \phi(g_1) * \phi(g_2)
	\end{align*}
	für alle $g_1,g_2 \in G$ und in diesem Fall ist
	\begin{align*}
		\Ker (\phi) := \phi^{-1}(\{1\})
	\end{align*}
	der Kern von $\phi$. \\
	Wir schreiben $\Hom(G,G')$ für die Menge der Gruppenhomomorphismen $\phi: G \to G'$.
\end{erinnerung}

\begin{bsp}
	Sei $n \in \nat$, $K$ ein Körper und $X$ eine Menge.
	\begin{enumerate}
		\item $\Sym(X)$, die symmetrische Gruppe aller Permutationen der Menge $X$ mit $f*g = g \circ f$, insbesondere $S_n:=\Sym(\{1, \dots, n\})$. Für $n \in \{1,2\}$ ist $S_n$ abelsch.
		\item $\integer$ und $\rest{n}:=\{ a + n\integer \, : a \in \integer \}$ mit der Addition.
		\item $\GL_n(K)$ mit der Matrizenmultiplikation. Spezialfall:
			\begin{align*}
				\GL_1(K) = K^{\times} = K \backslash \{0\}
			\end{align*}
		\item Für jeden Ring $R$ bilden die Einheiten $R^{\times}$ eine Gruppe unter Multiplikation, \\
		z.B. $\Mat_n(K)^{\times} = \GL_n(K)$ oder $\integer^{\times} = \mu_2=\{1,-1\}$
	\end{enumerate}
\end{bsp}

\begin{bsp}
	Ist $(G,*)$ eine Gruppe, so ist auch  $(G^{op}, *^{op})$ mit $G^{op} = G$ und $g *^{op} h = h * g$ eine Gruppe.
\end{bsp}

\begin{bem}
	Ist $G$ eine Gruppe und $h \in G$, so ist die Abbildung
	\begin{align*}
		\tau_h : \begin{cases}
		G \to G \\
		g \mapsto g*h \\
		\end{cases}
	\end{align*}
	eine Bijektion (also $\tau_h \in \Sym(G)$) mit Umkehrabbildung $\tau_{h^{-1}}$.
\end{bem}

\begin{satz}[vgl. LAAG I.3.8]
	Sei $G$ eine Gruppe. Zu jeder Teilmenge $X \subseteq G$ gibt es eine kleinste Untergruppe $\erz{X}$ von $G$, die $X$ enthält, nämlich 
	\begin{align*}
		\erz{X}= \bigcap \limits_{X \subseteq H \leq G} H
	\end{align*}
\end{satz}

\begin{bem}
	Man nennt $\erz{X}$ die von $X$ erzeugte Untergruppe $G$. Die Gruppe $G$ heißt endlich erzeugt, wenn $G = \erz{X}$ für eine endliche Menge $X \subseteq G$.\\
	Bsp.: $\integer = \erz{\{1\}}$
\end{bem}

\begin{satz}[vgl. LAAG II.2.8]
	Ein Gruppenhomomorphismus $\phi : G \to G'$ ist genau dann ein Isomorphismus, wenn	es einen Gruppenhomomorphismus $\phi': G' \to G$  mit $\phi' \circ \phi = \id_G$ und $\phi \circ \phi' = \id_{G'}$ gibt.
\end{satz}

\begin{bsp}
	Ist $G$ eine Gruppe, so bilden die Automorphismen $\Aut(G) \subseteq \Hom(G,G)$ eine Gruppe unter $\phi * \phi' = \phi' \circ \phi$. Ist $\phi \in \Aut(G)$ und $g \in G$ schreiben wir auch $g^{\phi}:=\phi(g)$.
\end{bsp}

\begin{satz}[vgl. LAAG III.2.14]
	Ein Gruppenhomomorphismus $\phi : G \to G'$ ist genau dann injektiv, wenn $\Ker(\phi)=1$.
\end{satz}

\begin{bsp}
	Seien $n \in \nat$ und $K$ ein Körper.
	\begin{enumerate}
		\item $\sgn: S_n \to \mu_2$ ist ein Gruppenhomomorphismus mit Kern die alternierende Gruppe $A_n$
		\item $\det: \GL_n(K) \to \korperx$ ist ein Gruppenhomomorphismus (vgl. Determinantenmultiplikationssatz) mit Kern $\SL_n(K)$
		\item $\pi_{n\integer}: \integer \to \rest{n}, a \mapsto a + n\integer$ ist ein Gruppenhomomorphismus mit Kern $n\integer$
		\item  Ist $A$ eine abelsche Gruppe, so ist
		\begin{align*}
			[n]: \begin{cases}
			A \to A \\
			x \mapsto n*x \\
			\end{cases}
		\end{align*}
		ein Gruppenhomomorphismus mit Kern $A[n]$, die $n$-Torsion von $A$, und Bild $nA$
		\item Ist $G$ eine Gruppe, so ist
		\begin{align*}
			\begin{cases}
			G \to G^{op} \\
			g \mapsto g^{-1} \\
			\end{cases}
		\end{align*}
		ein Isomorphismus (vgl. Übung)
	\end{enumerate}
\end{bsp}

\begin{defin}
	Seien $n,k \in \nat$. Für paarweise verschiedene Elemente $i_1, \dots , i_k \in \{1 , \dots , n\}$ bezeichnen wir mit $(i_1 \, \dots \, i_k)$ das $\sigma \in S_n$ gegeben durch
	\begin{align*}
		\sigma (i_j) &= i_{j+1} \quad \text{für } j=1 , \dots , k-1 \\
		\sigma (i_k) &= i_1 \\
		\sigma (i)  &= i \quad \text{für } i \in \{1,\dots,n\} \backslash \{i_1 , \dots , i_k \}
	\end{align*}
	Wir nennen $(i_1 \, \dots \, i_k)$ einen ($k$-)Zykel. \\
	Zwei Zykel $(i_1 \, \dots \, i_k)$ und $(j_1 \, \dots \, j_l) \in S_n$ heißen disjunkt, wenn
	\begin{align*}
		\{i_1 , \dots , i_k \} \cap \{j_1 , \dots , j_l \} = \emptyset
	\end{align*}
\end{defin}

\begin{satz}[LAAG IV.1.3]
	Jedes $\sigma \in S_n$ ist Produkt von Transpositionen (d.h. 2-Zyklen).
\end{satz}

\begin{lemma}
	Disjunkte Zykel kommutieren, d.h. sind $\tau_1, \tau_2 \in S_n$ disjunkte Zykel, so ist
	\begin{align*}
		\tau_1 \, \tau_2 = \tau_2 \, \tau_1
	\end{align*}.
\end{lemma}
\begin{proof}
	Sind $\tau_1 = ( i_1 \, \dots \, i_k)$ und $\tau_2 = (j_1 \, \dots \, j_l)$, so ist
	\begin{align*}
		\tau_1 \, \tau_2 (i) = \tau_2 \, \tau_1 (i) = \begin{cases}
		\tau_1 (i) & i \in \{i_1 , \dots , i_k \} \\
		\tau_2 (i) & i \in \{j_1 , \dots , j_l \} \\
		i & \text{sonst} \\
		\end{cases}
	\end{align*}
\end{proof}

\begin{satz} [Zykelzerlegung]
	Jedes $\sigma \in S_n$ ist ein Produkt von paarweise disjunkten $k$-Zyklen mit $k \geq 2$, eindeutig bis auf Reihenfolge (sogenannte Zykelzerlegung von $\sigma$)
\end{satz}
\begin{proof}
	Induktion nach $N:=\left| \{i \, : \sigma(i) = i \} \right|$ \textcolor{gray}{(sogenannter Stabilisator von $\sigma$)} \\
	(IA) $N=0$: $\qquad \sigma = \id$ \\
	(IS) $N > 0$: $\qquad$ Wähle $i_1$ mit $\sigma(i_1) \neq i_1$, betrachte $i_1, \sigma(i_1), \sigma^2(i_1), \dots$ Da $\{i_1 , \dots , n \}$ endlich ist und $\sigma$ bijektiv ist, existiert ein minimales $k \geq 2$ mit $\sigma^k(i_1)=i_1$. Setze $\tau_1 = (i_1 \, \sigma(i_1) \, \dots \, \sigma^{k-1}(i_1))$. Dann ist $\sigma = \tau_1 * \tau_1^{-1}\sigma$ und nach Induktionshypothese ist
	\begin{align*}
		\tau_1^{-1}\sigma = \tau_2 \cdots \tau_m
	\end{align*}
	Eindeutigkeit ist klar, denn jedes $i$ kann nur in dem Zykel $( i \: \sigma(i) \, \dots \, \sigma^{k-1}(i))$ vorkommen.
\end{proof}

\begin{bsp}
	Offensichtlich ist $(1 \, 2 \, 3 \, 4 \, 5) * ( 2\, 4)$ eine nicht-disjunkte Zerlegung in Zykel. Wir suchen daher eine solche Zykelzerlegung.
	\begin{align*}
		(1 \, 2 \, 3 \, 4 \, 5) * ( 2\, 4) &= (1 \, 4 \, 5) * (2 \, 3) \\
		&= (1 \, 4 \, 5) * (3 \, 2) \\
		&= (4 \, 5 \, 1) * (3 \, 2) \\
		&\neq (1 \, 5 \, 4) * (3 \, 2)
	\end{align*}
\end{bsp}