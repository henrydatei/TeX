\section{Normalteiler und Quotientengruppe}
%
Sei $G$ eine Gruppe.
%
\begin{defin}
	Eine Unterguppe $H \leq G$ ist \begriff{normal} (in Zeichen $H \normalteiler G$), wenn $g^{-1} h g \in H$ für alle $h \in H$ und $g \in G$. Ein \begriff{Normalteiler} von $G$ ist eine normale Untergruppe von $G$.
\end{defin}
%
\begin{bsp}
	\begin{enumerate}
		\item Ist $G$ abelsch, so ist jede Untergruppe von $G$ ein Normalteiler von $G$.
		\item Ist $\abb{\phi}{G}{H}$ ein Gruppenhomomorphismus, so ist $\Ker(\phi) \normalteiler G$. 
		\begin{align*}
			\phi(n)=1 \follows \phi(g^{-1} h g) = \phi(g)^{-1} * \phi(h) * \phi(g) = \phi(g)^{-1} * \phi(g) = 1 \quad \forall g \in G
		\end{align*}
		\item Jede Gruppe hat die trivialen Normalteiler $1 \normalteiler G$ und $G \normalteiler G$.
	\end{enumerate}
\end{bsp}
%
\begin{lemma}
	Seien $H \leq G$ und $N \normalteiler G$.
	\begin{enumerate}
		\item $H \normalteiler G \equivalent gH = HG$ für alle $g \in G$
		\item $HN = NH \quad , \quad HN \leq G \quad , \quad N \normalteiler HN \quad , \quad H \cap N \leq N \quad , \quad H \cap N \normalteiler H$
		\item Sind $N,H \normalteiler G$, so auch $H \cap N \normalteiler G$ und $HN \normalteiler G$.
		\item Für $g,g' \in G$ ist $gN * g'N = gg'N$.
	\end{enumerate}
\end{lemma}
\begin{proof}
	Wir beweisen die Eigenschaften unter Nutzung der Definition der Normalteiler.
	\begin{enumerate}
		\item \begin{itemize}
			\itemsep1pt
			\item [($\Rightarrow$)] $\forall g \in G \enspace \forall h \in H : g^{-1} h g \in H$ \\
			$\follows \forall g \in G : g^{-1}Hg \subseteq H$ \\
			$\follows \forall g \in G : Hg \subseteq gH \enspace , \enspace g^{-1}H \subseteq Hg^{-1}$ \\
			$\follows \forall g \in G : gH = Hg$.
			\item [($\Leftarrow$)] $\forall g \in G : gH = Hg$.\\
			$\follows \forall g \in G \forall h \in H \exists h' \in H : gh' = hg$ \\
			$\follows \forall g \in G \forall h \in H : g^{-1}hg = h' \in H$
			\end{itemize}
		\item \begin{itemize}
			\item $HN = \bigcup_{h \in H}{hN} \overset{(a)}{=} \bigcup_{h \in H}{Nh} = NH$
			\item $HN * HN = H* NH * N = H * HN * N = HN$ \\
			      $(HN)^{-1} = N^{-1} H^{-1} = NH = HN$ \\
			      $\follows HN \leq G$
			\item $N \normalteiler HN$: \checkmark
			\item $H \cap N \leq N$: \checkmark
			\item $H \cap N \normalteiler H$: $n \in H \cap N, h \in H \follows h^{-1}nh \in H \cap N$ \textcolor{gray}{(da $n$ normal in $G$)}
		\end{itemize}
		\item \begin{itemize}
			\item $H \cap N \normalteiler G$: $h \in H \cap N, g \in G \follows g^{-1}hg \in H \cap N$
			\item $HN \normalteiler G$: $g \in G \follows g *HN \overset{(a)}{=} Hg * N = H * gN \overset{(a)}{=} H * Ng = HNg$
		\end{itemize}
		\item $gN * g'N = g * Ng' * N = g * g'N * N = gg'N$
	\end{enumerate}
\end{proof}
%
\begin{satz}
	Sei $N \normalteiler G$. Dann ist $G/N$ mit dem Komplexprodukt als Verknüpfung eine Gruppe und $\abb{\pi_N}{G}{G/N}, g \mapsto gN$ ein Gruppenhomomorphismus mit $\Ker(\pi_N)=N$.
\end{satz}
\begin{proof}
	\begin{itemize}
		\item Komplexprodukt ist Verknüpfung auf $G/N$: vgl. 3.3(d)
		\item Gruppenaxiome übertragen sich von $G$ auf $G/N$ mit neutralem Element $1N$ und inversem Element $g^{-1}N$.
		\item $\pi_N$ ist Gruppenhomomorphismus: 3.3(d)
		\item $\Ker(\pi_N) = N$: 2.8(a)
	\end{itemize}
\end{proof}
%
\begin{kor}
	Die Normalteiler sind genau die Kerne von Gruppenhomomorphismen.
\end{kor}
%
\begin{defin}
	Für $N \normalteiler G$ heißt $G/N$ zusammen mit dem Komplexprodukt als Verknüpfung die \begriff{Quotientengruppe} (oder auch Faktorgruppe) von $G$ nach $N$ (oder auch $G$ modulo $N$).
\end{defin}
%
\begin{lemma}
	Sei $N \normalteiler G$. Für $H \leq G$ ist $\pi_N(H) = HN / H \leq G/N$ und $H \mapsto \pi_N(H)$ liefert eine Bijektion zwischen
	\begin{enumerate}
		\item den $H \leq G$ mit $N \leq H$, und
		\item $H \leq G/N$
	\end{enumerate}
\end{lemma}
\begin{proof}
	Wir zeigen die Untergruppeneigenschaft und die Bijektivität der Abbildung separat, letzteres durch Angabe der Umkehrabbildung.
	\begin{itemize}
		\item $\pi_N(H) = \menge{hN : h \in H} = \menge{hnN : h \in H, n \in N} = HN/H$
		\item Umkehrabbildung: $H \mapsto \pi_N^{-1}(H)$
		\begin{itemize}
			\item[] $H \leq G/N$: \hspace{5mm} $\pi_N(\pi_N^{-1}(H)) = H$, da $\pi_N$ surjektiv ist
			\item[] $N \leq H \leq G$: \hspace{1.5mm} $\pi_N^{-1}(\pi_N(H)) = \pi_N^{-1}(HN/N) = HN \subseteq HH = H$ 
		\end{itemize}
	\end{itemize}
\end{proof}
%
\begin{satz}[Homomorphiesatz]
	Sei $\abb{\phi}{G}{H}$ ein Gruppenhomomorphismus und $N \normalteiler G$ mit $N \subseteq \Ker(\phi)$. Dann existiert genau ein Gruppenhomomorphismus $\abb{\quer{\phi}}{G/N}{H}$ mit $\quer{\phi} \circ \pi_N = \phi$.
\end{satz}
\begin{proof}
	Existiert so ein $\quer{\phi}$, so ist $\quer{\phi}(gN)=(\quer{\phi} \circ \pi_N)(g) = \phi(G)$. Definiere $\quer{\phi}$ nun so.
	\begin{itemize}
		\item $\quer{\phi}$ ist wohldefiniert: \\
		$gN = g'N \overset{2.8}{\follows}$ ex. $g' = gn$ für ein $n \in \nat \follows \phi(g') = \phi(g) * \phi(n) = \phi(g)$
		\item $\quer{\phi}$ ist Homomorphismus: \\
		$\quer{\phi}(gN * g'N) = \quer{\phi}(gg'N) = \phi(gg') = \phi(g) \phi(g') = \quer{\phi}(gN) * \quer{\phi}(g'N)$
	\end{itemize}
\end{proof}
%
\begin{kor}
	Ein Gruppenhomomorphismus $\abb{\phi}{G}{H}$ liefert einen Isomorphismus
	\begin{align*}
		\abb{\quer{\phi}}{G/\Ker(\phi)}{\bild(\phi) \leq H}
	\end{align*}
\end{kor}
%
\begin{kor}[1. Noetherscher Isomorphiesatz]
	Seien $H \leq G$ und $N \normalteiler G$. Der Homomorphismus
	\begin{align*}
		\abb{\phi}{H \overset{\iota}{\hookrightarrow} HN}{HN/N} 
	\end{align*}
	induziert einen Isomorphismus
	\begin{align*}
		\abb{\quer{\phi}}{H/(H \cap N)}{HN/N}
	\end{align*}
\end{kor}
\begin{proof}
	\begin{itemize}
		$\phi$ surjektiv, denn für $h \in H, n \in N$ ist $hnN=hN=\phi(n) \in \phi(H) = \bild(\phi)$. Außerdem ist $\Ker(\phi) = H \cap \Ker(\pi_N) = H \cap N$.
	\end{itemize}
\end{proof}
%
\begin{kor}
	Seien $N \normalteiler G$ und $N \leq H \normalteiler G$. Der Homomorphismus
	\begin{align*}
		\abb{\pi_N}{G}{G/H}
	\end{align*}
	induziert einen Isomorphismus
	\begin{align*}
		(G/N)/(H/N)) \overset{\isomorph}{\longrightarrow} G/H 
	\end{align*}
\end{kor}
\begin{proof}
	Da $N \leq H$ liefer $\pi_H$ einen Epimorphismus $\abb{\quer{\pi_N}}{G/N}{G/H}$ (vgl. 3.8). Dieser hat $\Ker(\quer{\pi_H}) = H/N$ und induziert nach 3.9 einen Isomorphismus 
	\begin{align*}
		(G/N)/\Ker(\quer{\pi_H}) \overset{\isomorph}{\longrightarrow} \bild(\quer{\pi_H}) = G/H
	\end{align*}
\end{proof}
%
\begin{defin}
	Seien $x,x',g \in G$ und $H,H' \leq G$.
	\begin{enumerate}
		\item $x^g := g^{-1}xg$ ist die \begriff{Konjugation} von $x$ mit $g$.
		\item $x$ und $x^{-1}$ sind \begriff{konjugiert} (in G) $\enspace :\Leftrightarrow \enspace \exists g \in G : x' = x^g$
		\item $H$ und $H'$ heißen \begriff{konjugiert} (in G) $\enspace : \Leftrightarrow \enspace \exists g \in G : H' = H^g := \menge{h^g : h \in H}$
	\end{enumerate}
\end{defin}
%
\begin{lemma}
	Die Abbildung
	\begin{align*}
		int: \begin{cases}
		G \to \Aut(G) \\ g \mapsto (x \mapsto x^g)
		\end{cases}
	\end{align*}
	ist ein Gruppenhomomorphismus.
\end{lemma}
\begin{proof}
	\begin{itemize}
		\item $int(g) \in \Hom(G,G)$: \hspace{5mm} $(xy)^g=g^{-1}xyg=g^{-1}xgg^{-1}yg = x^g * y^g \enspace \forall x,y,g \in  G$
		\Item \begin{align} \label{konjugation_additiv}%
			(x^g)^h = h^{-1} x^g h = h^{-1}g^{-1} x gh = (gh)^{-1} x (gh) = x^{gh}%
		\end{align}
		\item $int(g) \in \Aut(G)$: \hspace{5mm} Umkehrabbildung zu $int(g)$ ist $int(g^{-1})$
		\item $int \in \Hom(G,\Aut(G))$: \hspace{5mm} $int(gh) \overset{(\ref{konjugation_additiv})}{=} int(h) \circ int(g) = int(g) * int(h)$
	\end{itemize}
\end{proof}
%
\begin{defin}
	\begin{enumerate}
		\item $\Inn(G) := \bild(int) \leq \Aut(G)$ Gruppe der \begriff{inneren Automorphismen} von $G$
		\item $\zentrum(G) := \ker(int) = \menge{g \in G : gx = xg \enspace \forall x \in G}$ das \begriff{Zentrum} von $G$
		\item $H \leq G$ ist charakteristisch $\enspace :\Leftrightarrow \forall \sigma \in \Aut(G): H = H^{\sigma} = \menge{h^{\sigma} : h \in H}$
	\end{enumerate}
\end{defin}
%
\begin{bem}
	\begin{itemize}
		\item Konjugiertheit ist eine Äquivalenzrelation \textcolor{gray}{(auf $G$ oder Menge der Untergruppen von $G$)}
		\item $H \leq G$ ist normal $\equivalent H = H^{\sigma} \enspace \forall \sigma \in \Inn(G)$
		\item Deshalb gilt für $H \leq G$: $H$ ist charakteristisch $\follows H$ ist normal
	\end{itemize}
\end{bem}
%
\begin{bsp}
	$\zentrum(G)$ ist charakteristisch in $G$.
\end{bsp}