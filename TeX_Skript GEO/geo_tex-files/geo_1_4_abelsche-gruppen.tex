\section{Abelsche Gruppen}
%
Sei $G$ eine Gruppe.
%
\begin{defin}
	$G$ ist \begriff{zyklisch} $\equivalent G = \erz{g}$ für ein $g \in G$.
\end{defin}
%
\begin{bsp}
	\begin{enumerate}
		\item $\integer$ ist zyklisch.
		\item $\rest{n}$ ist zyklisch der Ordnung $n$.
		\item $C_n = \erz{(1 \, 2 \, \cdots \, n)} \leq S_n$ ist zyklisch der Ordnung $n$.
		\item Ist $\#G = p$ eine Primzahl, so ist $G$ zyklisch (vgl. Ü6).
	\end{enumerate}
\end{bsp}
%
\begin{lemma} [!]
	Die Untergruppen von $(\integer, +)$ sind genau die $\erz{k} = k\integer$ mit $k \in \nat_0$ und für $k_1, \dots, k_r \in \integer$ ist $\erz{k_1, \dots , k_r} = \erz{k}$ mit $\ggT(k_1, \dots , k_r) = k$.
\end{lemma}
\begin{proof}
	Jede Untergruppe von $\integer$ ist ein Ideal von $(\integer,+,*)$ und $\integer$ ist Hauptidealring.
\end{proof}
\begin{proof}
	Sei $H \leq \integer$. Setze $k = \min (H \cap N)$, o.E. sei $H \neq \menge{0}$. Offensichtlich gilt $\erz{k} \subseteq H$.
	\begin{itemize}
		\item $n \in H \follows n = q*k + r$ mit $q,r \in \integer, 0 \leq r < k \follows r = n - \underbrace{g*k}_{k+\cdots+k}$ \\
		$\follows r = 0 \text{wegen der Minimalität von } k$ \\
		$\follows n = q*k$, d.h. $n \in \erz{k}$
		\item $k = \ggT(k_1, \dots , k_r)$: \\
		$k_i \in \erz{k} \follows k \teilt k_i \enspace \forall i$ \\
		$k \in \erz{k_1, \dots , k_r} \follows k = n_1 k_1 + \cdots + n_r k_r$ mit $n_i \in \integer$ \\
		$d \teilt k_i \enspace \forall i \follows d \teilt k$
	\end{itemize}
\end{proof}
%
\begin{satz}
	Sei $G = \erz{g}$ zyklisch. Dann ist $G$ abelsch und $G \isomorph (\integer,+)$ oder $G \isomorph (\rest{n}, +)$ mit $n = \# G < \infty$.
\end{satz}
\begin{proof}
	Betrachte
	\begin{align*}
		\phi \colon \begin{cases} \integer \to G \\ k \mapsto g^k \end{cases}
	\end{align*}
	$\phi$ ist Homomorphismus und surjektiv, da $G = \erz{g}$. Nach 3.9. ist $G = \bild(\phi) \isomorph \integer / \Ker(\phi)$. Nach 4.3 ist $\Ker(\phi)  = \erz{n}$ für ein $n \in \nat_0$. Ist $n=0$, so ist $\Ker(\phi) = \menge{0}$ und $G \isomorph \integer$. Ist $n > 0$, so ist $G \isomorph \rest{n}$ und $n = \# \rest{n} = \# G$.
\end{proof}
%
\begin{satz}
	Sei $(G,+) = \erz{g}$ zyklisch der Ordnung $n \in \nat$.
	\begin{enumerate}
		\item Zu jedem $d \in \nat$ mit $d \teilt n$ hat $G$ genau eine Unterguppe der Ordnung $d$, nämlich \\
		$U_d := \erz{\frac{n}{d} \enspace g}$. \textcolor{gray}{Damit ist jede Untergruppe einer zyklischen Gruppe wieder zyklisch.}
		\item Für $d \teilt h$ und $d' \teilt h'$ ist $U_d \subseteq U_{d'} \follows d \teilt d'$.
		\item Für $k_1 , \dots k_r \in \integer$ ist $\erz{k_1g, \dots , k_r g} = \erz{eg} = U_{\frac{n}{e}}$ mit $e = \ggT(k_1, \dots , k_r , n)$.
		\item Für $k \in \integer$ ist $\ord(kg) = \frac{n}{\ggT(k,n)}$.
	\end{enumerate}
\end{satz}
\begin{proof}
	Betrachte wieder $\abb{\phi}{\integer}{G}, k \mapsto kg$.
	\begin{enumerate}
		\item Nach 3.7 und 4.3 liefert $\phi$ eine Bijektion $\menge{e \in \integer : n\integer \leq e\integer} \to \menge{H \leq G}$ und \\
		$n\integer \leq e\integer \equivalent e \teilt n$. \\
		Ist $H = \phi(e\integer) = \erz{eg}$, so ist $H \isomorph e\integer / n\integer$ \textcolor{gray}{($n\integer = \ker(\phi)$)}, also
		\begin{align*}
			n = (\integer : n \integer) = (\integer : e\integer) (e\integer : n \integer) = e * \# H
		\end{align*}
		\item $U_d \subseteq U_{d'} \equivalent \erz{\frac{n}{d} \enspace g} \subseteq \erz{\frac{n}{d'} \enspace g} \equivalent \frac{n}{d} \integer \leq \frac{n}{d'} \integer \follows \frac{n}{d'} \teilt \frac{n}{d} \equivalent d \teilt d'$
		\item Mit $H = \erz{k_1 , \dots , k_r} \leq \integer$ ist $n\integer \leq H$, $\phi(H) = \erz{k_1 g , \dots , k_r g}$ \textcolor{gray}{($n \in \ker(\phi)$)}. \\
		Nach 4.3 ist $H = \erz{e}$ mit $e = \ggT(k_1 , \dots , k_r , n)$ und somit $\erz{k_1 g , \dots k_r g} = \phi(e\integer) = U_{\frac{n}{e}}$.
		\item $\ord(kg) = \# \erz{kg} \overset{(c)}{=} U_{\frac{n}{e}}$ mit $e = \ggT(k,n)$. \textcolor{gray}{(Fall (c) mit $r=1$)}
	\end{enumerate}
\end{proof}
%
\begin{lemma}
	Seien $a,b \in G$. Kommutieren $a$ und $b$ und sind $\ord(a), \ord(b)$ teilerfremd, so ist \\
	$\ord(a*b) = \ord(a) * \ord(b)$.
\end{lemma}
\begin{proof}
	Nach 2.12 ist $\erz{a} \cap \erz{b} = \menge{1}$. Ist $(ab)^k = a^k * b^k$, so ist $a^k = b^{-k} \in \erz{a} \cap \erz{b} = 1$, also $a^k = b^k = 1$. Somit ist $(ab)^k = 1 \equivalent a^k = 1$ und $b^k = 1$, und damit
	\begin{align*}
		\ord(ab) = \kgV(\ord(a),\ord(b)) = \ord(a) * \ord(b)
	\end{align*}
\end{proof}
%
\begin{kor}
	Ist $G$ abelsch und sind $a,b \in G$ mit $\ord(a) = < \infty, \ord(b) = n < \infty$, so existiert ein $c \in G$ mit $\ord(c) = \kgV(\ord(a), \ord(b))$.
\end{kor}
\begin{proof}
	Schreibe $m = m_0 * m'$ und $n = n_0 * n'$ mit $m_0 * n_0 = \kgV(m,n)$ und $\ggT(m_0,n_0) = 1$. \\
	$\follows \ord(a^{m'}) = m_0 \enspace , \enspace \ord(b^{n'}) = n_0$ \\
	$\follows \ord(a^{m'}b^{n'}) \overset{4.6}{=} m_0*n_0 = \kgV(m,n)$ \\
	Dann ist $c:= a^{m'}b^{n'}$.
\end{proof}
%
\begin{thm}[Struktursatz für endlich erzeugte abelsche Gruppen]
	Jede endlich erzeugte abelsche Gruppe $G$ ist eine direkte Summe zyklischer Gruppen
	\begin{align*}
		G^r \isomorph \integer^r \oplus \bigoplus\limits_{i=1}^{k}{\rest{d_i}}
	\end{align*}
	mit eindeutig bestimmten $d_i, \dots, d_k > 1$, die $d_i \teilt d_{i+1}$ für alle $i$ erfüllen.
\end{thm}
\begin{proof}
	Die Existenz folgt aus LAAG VIII.6.14 (Hauptsatz über endlich erzeugte Moduln über Hauptidealringen). \\
	Eindeutigkeit: Für $d \in \nat$ ist $\# G/dG = \# (\rest{d})^r \oplus \bigoplus_{i=1}^{k}{(\rest{d_i})/d*\rest{d_i}} \overset{4.5(d)}{=} d^r * \prod_{i=1}^{k}{\frac{d_i}{\ggT(d,d_i)}}$. Daraus kann man nun $r,k,d_1, \dots, d_k$ erhalten, z.B. für $p$ prim $p \teilt nicht d_i \enspace \forall i$ ist $\# G/pG = p^r * \prod_{i=1}^{k}{d_i}$.
\end{proof}
%
\begin{lemma}
	Sei $G = (G,+) = \erz{g}$ zyklisch der Ordnung $n \in  \nat$. Die Endomorphismen von $G$ sind genau die 
	\begin{align*}
		\phi_{\quer{k}}: \begin{cases}
		G \to G \\
		x \mapsto kx
		\end{cases}
	\end{align*}
	für $\quer{k} = k +n\integer \in \rest{n}$. Dabei ist $\phi_{\quer{l}} \circ \phi_{\quer{k}} = \phi_{\quer{kl}}$ für $\quer{k}, \quer{l} \in \rest{n}$.
\end{lemma}
\begin{proof}
	Zu zeigen sind eine Reihe von Aussagen.
	\begin{itemize}
		\item $\phi_{\quer{k}}$ wohldefiniert: $\quer{k_1} = \quer{k_2} \follows k_2 = k_1 + an$ mit $a \in \integer$. Dann ist auch $k_2 x = k_1 x + a*nx = k_1 x \enspace \forall x \in G$.
		\item $\phi_{\quer{k}} \in \Hom(G,G)$: klar, da $G$ abelsch.
		\item $\quer{k} = \quer{l} \equivalent \phi_{\quer{k}} = \phi_{\quer{l}}$: $\quad$ \textcolor{gray}{(zeige $\Leftarrow$; $\Rightarrow$ ist Wohldefiniertheit)} \\
		$\phi_{\quer{k}} = \phi_{\quer{l}} \follows \phi_{\quer{k}} (g) = \phi_{\quer{l}} (g) \follows (k-l) g = 0 \overset{\ord(g)=n}{\follows} n \teilt k-l \follows \quer{k} = \quer{l}$
		\item $\phi \in \Hom(G,G)$: $\follows \phi = \phi_{\quer{k}}$ für ein $k \in \integer$ ; $\phi(g) = k*g$ für ein $k \in \integer \follows \phi = \phi_{\quer{k}}$
		\item $\phi_{\quer{l}} \circ \phi_{\quer{k}} = \phi_{\quer{kl}}$: $\quad l*(k*x) = (l*k) * x$ \checkmark
	\end{itemize}
\end{proof}
%
\begin{satz}
	Ist $G$ zyklisch von Ordnung $n \in \nat$, so ist $\Aut(G) \isomorph (\rest{n})^{\times}$. \textcolor{gray}{(multiplikativ)}
\end{satz}
\begin{proof}
	$\Aut(G) \subseteq \Hom(G,G) = \menge{\phi_{\quer{k}} : \quer{k} \in \rest{n}}$. \\
	\begin{align*}
		\phi_{\quer{k}} \in \Aut(G) &\equivalent \exists \quer{l} \in \rest{n}: \phi_{\quer{l}} \circ \phi_{\quer{k}} = \phi_{\quer{1}} \\
		&\equivalent \exists \quer{l} \in \rest{n}: \quer{l} * \quer{k} = 1 \\
		&\equivalent \quer{k} \in \einheit{(\rest{n})}
	\end{align*}
	und die Abbildung $\einheit{(\rest{n})} \to \Aut(G)$ mit $\quer{k} \mapsto \phi_{\quer{k}}$ ist ein Isomorphismus. \textcolor{gray}{Offensichtlich ist diese ein Homomorphismus und die Bijektivität folgt aus der Tatsache, dass jeder Endomorphismus genau diese Gestalt $\phi_{\quer{k}}$ hat.}
\end{proof}
%
\begin{defin}
	Die Abbildung $\abb{\Phi}{\nat}{\nat}$ gegeben durch
	\begin{align*}
		\Phi(n) = \# \einheit{(\rest{n})}
	\end{align*}
	ist die \begriff{Euler'sche Phi-Funktion}.
\end{defin}
%
\begin{bsp}
	Ist $p$ prim, so ist $\Phi(p) = p-1$, da $\rest{p}$ ein Körper ist.
\end{bsp}