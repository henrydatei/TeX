\documentclass[a4paper, 11pt, chapterprefix, ngerman]{scrreprt}
\KOMAoptions{%
	parskip=half,%
	fontsize=11pt}

%=== PACKAGES ===================================================
\input{../tex_packages}

%================================================================
%=== SETTINGS ===================================================
%================================================================

% |--------------------------|
% | Farben                   |
% |--------------------------|
\definecolor{lightgrey}{gray}{0.89}
\definecolor{darkgrey}{gray}{0.6}

\definecolor{blue}{rgb}{50,85,150}
\definecolor{lightred}{rgb}{1,0.6,0.6}
\definecolor{darkgreen}{rgb}{0,0.6,0}

% |--------------------------|
% | Zähler                   |
% |--------------------------|
\newcounter{themcount}
\newcounter{defcount}
%\renewcommand{\thethemcount}{\relax}

\numberwithin{themcount}{section}
\counterwithout{section}{chapter}

% |--------------------------|
% | Mathematik               |
% |--------------------------|
%============= BUCHSTABEN =============
\renewcommand{\epsilon}{\varepsilon}
\renewcommand{\phi}{\varphi}

%============= LOGIK =============
\newcommand{\follows}{\quad \Rightarrow \quad}
\newcommand{\equivalent}{\quad \Leftrightarrow \quad}

%============= MENGEN =============
\newcommand{\nat}{\mathbb{N}}				% natürliche Zahlen
\newcommand{\integer}{\mathbb{Z}}			% ganze Zahlen
\newcommand{\real}{\mathbb{R}}				% reelle Zahlen
\newcommand{\complex}{\mathbb{C}}			% komplexe Zahlen
\newcommand{\korper}{\mathbb{K}}			% Körper IR oder IC
\newcommand{\ringx}{R^{\times}}				% Einheiten Ring
\newcommand{\korperx}{K^{\times}}			% Einheiten Körper
\newcommand{\rest}[1]{\integer / #1 \integer} % Restklassen Z/( )Z
\newcommand{\ohneNull}{\backslash \menge{0}}% Menge ohne Null \{0}
\newcommand{\einheit}[1]{{#1}^{\times}}

\newcommand{\menge}[1]{\{ #1 \}}
\newcommand{\card}[1]{\abs{#1}}
\newcommand{\st}{\,|\,}
\DeclareMathOperator{\inn}{int} % Set of inner points
\DeclareMathOperator{\ext}{ext} % Set of outer points
\DeclareMathOperator{\cl}{cl} 	% Abschluss
\DeclareMathOperator{\diam}{diam}

%============= FUNKTIONEN =============
\newcommand{\abb}[3]{#1 \colon #2 \to #3}
\DeclareMathOperator{\id}{id}
\DeclareMathOperator{\graph}{graph}
\DeclareMathOperator{\grad}{grad}
\newcommand{\quer}[1]{\bar{#1}}

%============= ZAHLENTHEORIE =============
\newcommand{\teilt}{\mid}
\newcommand{\normalteiler}{\trianglelefteq}
\newcommand{\nichtnormal}{\ntrianglelefteq}
\DeclareMathOperator{\ggT}{ggT}
\DeclareMathOperator{\kgV}{kgV}

\newcommand{\simfrac}[1]{\frac{1}{#1}}

%============= VEKTORRÄUME =============
\newcommand{\basisB}{\mathcal{B}}
\newcommand{\darstMat}[2]{M_{\mathcal{#1}} (#2)}
\newcommand{\MBf}{\darstMat{\basisB}{f}}
\newcommand{\one}{\mathbbm{1}}

\newcommand{\isomorph}{\cong}

\newcommand{\erz}[1]{\langle #1 \rangle}
\newcommand{\scal}[2]{\langle #1\, ,\, #2 \rangle}

\DeclareMathOperator{\Span}{span}

\DeclareMathOperator{\Hom}{Hom}
\DeclareMathOperator{\End}{End}
\DeclareMathOperator{\Aut}{Aut}
\DeclareMathOperator{\Ker}{Ker}
\DeclareMathOperator{\bild}{Im}

\DeclareMathOperator{\Mat}{Mat}
\DeclareMathOperator{\diag}{diag}
\DeclareMathOperator{\GL}{GL}
\DeclareMathOperator{\SL}{SL}
\DeclareMathOperator{\rang}{rang}
\DeclareMathOperator{\tr}{tr}
\newcommand{\transpose}[1]{\left( #1 \right)^{\top}}

\DeclareMathOperator{\sgn}{sgn}
\DeclareMathOperator{\Sym}{Sym}

\DeclareMathOperator{\Eig}{Eig}

%============= GRUPPEN =============
\DeclareMathOperator{\ord}{ord}
\DeclareMathOperator{\Inn}{Inn}
\DeclareMathOperator{\zentrum}{Z}
\DeclareMathOperator{\fix}{Fix}
\DeclareMathOperator{\stab}{Stab}

%============= RINGE =============

%============= METRISCHE RÄUME =============
\newcommand{\abs}[1]{\left| #1 \right|}
\newcommand{\norm}[1]{\left\lVert #1 \right\rVert}

%============= DIFFERENZIERUNG =============
\newcommand{\partdiff}[1]{\frac{\partial}{\partial #1 }}
\newcommand{\diff}[1]{\enspace \mathrm{d}#1}
\newcommand{\dx}{\diff{x}}
\newcommand{\dy}{\diff{y}}
\newcommand{\da}{\diff{a}}

\DeclareMathOperator{\Lin}{L}		% Menge der lineare Abb.
\DeclareMathOperator{\cdiff}{C}		% Menge der stetig-diffbaren Abb.

%============= INTEGRATION =============
\DeclareMathOperator{\tangentialraum}{T}
\DeclareMathOperator{\normalenraum}{N}
\DeclareMathOperator{\divergenz}{div}
\renewcommand{\div}{\divergenz}
\DeclareMathOperator{\rot}{rot}
\newcommand{\rand}{\partial}

\newcommand{\bigabb}[5]{#1 \colon \left\lbrace%
\begin{array}{ccl}%
	#2 & \to & #3 \\%
	#4 & \mapsto & #5%
\end{array}%
\right.}
%
\newcommand{\bigabbnoname}[4]{\left\lbrace%
\begin{array}{ccl}%
	#1 & \to & #2 \\%
	#3 & \mapsto & #4%
\end{array}%
\right.}

\input{../tex_theoreme}

\DeclareMathSymbol{*}{\mathbin}{symbols}{"01}

% |--------------------------|
% | neue Befehle             |
% |--------------------------|
\renewcommand{\thefootnote}{\alph{footnote}}

\renewcommand*{\theequation}{\arabic{equation}}
\renewcommand{\labelitemi}{$\vartriangleright$}
\renewcommand{\labelenumi}{\alph{enumi})}

%$\blacktriangleright \vartriangleright \triangleright$ hier steht ein text

%Aufzählungszeichen für Formel in der gleichen Zeile
\def\Item{\item~\vspace{-1\normalbaselineskip}}

\begin{document}

\setlength\abovedisplayshortskip{3pt}
\setlength\belowdisplayshortskip{3pt}
\setlength\abovedisplayskip{3pt}
\setlength\belowdisplayskip{3pt}

\title{Geometrie}
\author{Eric Kunze}
\date{\today}
\maketitle[0]

\chapter{Endliche Gruppen}

\section{Erinnerung und Beispiele}

\begin{erinnerung}
	Eine Gruppe ist ein Paar $(G,\star)$ bestehend aus einer Menge $G$ und einer Abbildung $\star : G \times G \to G$, das die Axiome Assoziativität, Existenz eines neutralen Elements und Existenz eines inversen Elements erfüllt. Wir schreiben auch $G$ für $(G,\star)$. Die Gruppe ist abelsch, wenn $g \star h = h \star g$ für alle $g,h \in G$ gilt. Eine allgemeine Gruppe schreiben wir multiplikativ mit neutralem Element $1$, abelsche Gruppen auch additiv mit neutralem Element $0$. \\
	Eine Teilmenge $H \subseteq G$ ist eine Untergruppe von $G$, in Zeichen $H \leq G$, wenn $H \neq \emptyset$ und $H$ abgeschlossen ist unter der Verknüpfung und dem Bilden von Inversen. \\
	Wir schreiben $1$ (bzw. $0$) für die triviale Untergruppe $\{1\}$ (bzw. $\{0\}$) von $G$. \\
	Eine Abbildung $\phi : G \to G'$ zwischen Gruppen ist ein Gruppenhomomorphismus, wenn
	\begin{align*}
		\phi(g_1 * g_2) = \phi(g_1) * \phi(g_2)
	\end{align*}
	für alle $g_1,g_2 \in G$ und in diesem Fall ist
	\begin{align*}
		\Ker (\phi) := \phi^{-1}(\{1\})
	\end{align*}
	der Kern von $\phi$. \\
	Wir schreiben $\Hom(G,G')$ für die Menge der Gruppenhomomorphismen $\phi: G \to G'$.
\end{erinnerung}

\begin{bsp}
	Sei $n \in \nat$, $K$ ein Körper und $X$ eine Menge.
	\begin{enumerate}
		\item $\Sym(X)$, die symmetrische Gruppe aller Permutationen der Menge $X$ mit $f*g = g \circ f$, insbesondere $S_n:=\Sym(\{1, \dots, n\})$. Für $n \in \{1,2\}$ ist $S_n$ abelsch.
		\item $\integer$ und $\rest{n}:=\{ a + n\integer \, : a \in \integer \}$ mit der Addition.
		\item $\GL_n(K)$ mit der Matrizenmultiplikation. Spezialfall:
			\begin{align*}
				\GL_1(K) = K^{\times} = K \backslash \{0\}
			\end{align*}
		\item Für jeden Ring $R$ bilden die Einheiten $R^{\times}$ eine Gruppe unter Multiplikation, \\
		z.B. $\Mat_n(K)^{\times} = \GL_n(K)$ oder $\integer^{\times} = \mu_2=\{1,-1\}$
	\end{enumerate}
\end{bsp}

\begin{bsp}
	Ist $(G,*)$ eine Gruppe, so ist auch  $(G^{op}, *^{op})$ mit $G^{op} = G$ und $g *^{op} h = h * g$ eine Gruppe.
\end{bsp}

\begin{bem}
	Ist $G$ eine Gruppe und $h \in G$, so ist die Abbildung
	\begin{align*}
		\tau_h : \begin{cases}
		G \to G \\
		g \mapsto g*h \\
		\end{cases}
	\end{align*}
	eine Bijektion (also $\tau_h \in \Sym(G)$) mit Umkehrabbildung $\tau_{h^{-1}}$.
\end{bem}

\begin{satz}[vgl. LAAG I.3.8]
	Sei $G$ eine Gruppe. Zu jeder Teilmenge $X \subseteq G$ gibt es eine kleinste Untergruppe $\erz{X}$ von $G$, die $X$ enthält, nämlich 
	\begin{align*}
		\erz{X}= \bigcap \limits_{X \subseteq H \leq G} H
	\end{align*}
\end{satz}

\begin{bem}
	Man nennt $\erz{X}$ die von $X$ erzeugte Untergruppe $G$. Die Gruppe $G$ heißt endlich erzeugt, wenn $G = \erz{X}$ für eine endliche Menge $X \subseteq G$.\\
	Bsp.: $\integer = \erz{\{1\}}$
\end{bem}

\begin{satz}[vgl. LAAG II.2.8]
	Ein Gruppenhomomorphismus $\phi : G \to G'$ ist genau dann ein Isomorphismus, wenn	es einen Gruppenhomomorphismus $\phi': G' \to G$  mit $\phi' \circ \phi = \id_G$ und $\phi \circ \phi' = \id_{G'}$ gibt.
\end{satz}

\begin{bsp}
	Ist $G$ eine Gruppe, so bilden die Automorphismen $\Aut(G) \subseteq \Hom(G,G)$ eine Gruppe unter $\phi * \phi' = \phi' \circ \phi$. Ist $\phi \in \Aut(G)$ und $g \in G$ schreiben wir auch $g^{\phi}:=\phi(g)$.
\end{bsp}

\begin{satz}[vgl. LAAG III.2.14]
	Ein Gruppenhomomorphismus $\phi : G \to G'$ ist genau dann injektiv, wenn $\Ker(\phi)=1$.
\end{satz}

\begin{bsp}
	Seien $n \in \nat$ und $K$ ein Körper.
	\begin{enumerate}
		\item $\sgn: S_n \to \mu_2$ ist ein Gruppenhomomorphismus mit Kern die alternierende Gruppe $A_n$
		\item $\det: \GL_n(K) \to \korperx$ ist ein Gruppenhomomorphismus (vgl. Determinantenmultiplikationssatz) mit Kern $\SL_n(K)$
		\item $\pi_{n\integer}: \integer \to \rest{n}, a \mapsto a + n\integer$ ist ein Gruppenhomomorphismus mit Kern $n\integer$
		\item  Ist $A$ eine abelsche Gruppe, so ist
		\begin{align*}
			[n]: \begin{cases}
			A \to A \\
			x \mapsto n*x \\
			\end{cases}
		\end{align*}
		ein Gruppenhomomorphismus mit Kern $A[n]$, die $n$-Torsion von $A$, und Bild $nA$
		\item Ist $G$ eine Gruppe, so ist
		\begin{align*}
			\begin{cases}
			G \to G^{op} \\
			g \mapsto g^{-1} \\
			\end{cases}
		\end{align*}
		ein Isomorphismus (vgl. Übung)
	\end{enumerate}
\end{bsp}

\begin{defin}
	Seien $n,k \in \nat$. Für paarweise verschiedene Elemente $i_1, \dots , i_k \in \{1 , \dots , n\}$ bezeichnen wir mit $(i_1 \, \dots \, i_k)$ das $\sigma \in S_n$ gegeben durch
	\begin{align*}
		\sigma (i_j) &= i_{j+1} \quad \text{für } j=1 , \dots , k-1 \\
		\sigma (i_k) &= i_1 \\
		\sigma (i)  &= i \quad \text{für } i \in \{1,\dots,n\} \backslash \{i_1 , \dots , i_k \}
	\end{align*}
	Wir nennen $(i_1 \, \dots \, i_k)$ einen ($k$-)Zykel. \\
	Zwei Zykel $(i_1 \, \dots \, i_k)$ und $(j_1 \, \dots \, j_l) \in S_n$ heißen disjunkt, wenn
	\begin{align*}
		\{i_1 , \dots , i_k \} \cap \{j_1 , \dots , j_l \} = \emptyset
	\end{align*}
\end{defin}

\begin{satz}[LAAG IV.1.3]
	Jedes $\sigma \in S_n$ ist Produkt von Transpositionen (d.h. 2-Zyklen).
\end{satz}

\begin{lemma}
	Disjunkte Zykel kommutieren, d.h. sind $\tau_1, \tau_2 \in S_n$ disjunkte Zykel, so ist
	\begin{align*}
		\tau_1 \, \tau_2 = \tau_2 \, \tau_1
	\end{align*}.
\end{lemma}
\begin{proof}
	Sind $\tau_1 = ( i_1 \, \dots \, i_k)$ und $\tau_2 = (j_1 \, \dots \, j_l)$, so ist
	\begin{align*}
		\tau_1 \, \tau_2 (i) = \tau_2 \, \tau_1 (i) = \begin{cases}
		\tau_1 (i) & i \in \{i_1 , \dots , i_k \} \\
		\tau_2 (i) & i \in \{j_1 , \dots , j_l \} \\
		i & \text{sonst} \\
		\end{cases}
	\end{align*}
\end{proof}

\begin{satz} [Zykelzerlegung]
	Jedes $\sigma \in S_n$ ist ein Produkt von paarweise disjunkten $k$-Zyklen mit $k \geq 2$, eindeutig bis auf Reihenfolge (sogenannte Zykelzerlegung von $\sigma$)
\end{satz}
\begin{proof}
	Induktion nach $N:=\left| \{i \, : \sigma(i) = i \} \right|$ \textcolor{gray}{(sogenannter Stabilisator von $\sigma$)} \\
	(IA) $N=0$: $\qquad \sigma = \id$ \\
	(IS) $N > 0$: $\qquad$ Wähle $i_1$ mit $\sigma(i_1) \neq i_1$, betrachte $i_1, \sigma(i_1), \sigma^2(i_1), \dots$ Da $\{i_1 , \dots , n \}$ endlich ist und $\sigma$ bijektiv ist, existiert ein minimales $k \geq 2$ mit $\sigma^k(i_1)=i_1$. Setze $\tau_1 = (i_1 \, \sigma(i_1) \, \dots \, \sigma^{k-1}(i_1))$. Dann ist $\sigma = \tau_1 * \tau_1^{-1}\sigma$ und nach Induktionshypothese ist
	\begin{align*}
		\tau_1^{-1}\sigma = \tau_2 \cdots \tau_m
	\end{align*}
	Eindeutigkeit ist klar, denn jedes $i$ kann nur in dem Zykel $( i \: \sigma(i) \, \dots \, \sigma^{k-1}(i))$ vorkommen.
\end{proof}

\begin{bsp}
	Offensichtlich ist $(1 \, 2 \, 3 \, 4 \, 5) * ( 2\, 4)$ eine nicht-disjunkte Zerlegung in Zykel. Wir suchen daher eine solche Zykelzerlegung.
	\begin{align*}
		(1 \, 2 \, 3 \, 4 \, 5) * ( 2\, 4) &= (1 \, 4 \, 5) * (2 \, 3) \\
		&= (1 \, 4 \, 5) * (3 \, 2) \\
		&= (4 \, 5 \, 1) * (3 \, 2) \\
		&\neq (1 \, 5 \, 4) * (3 \, 2)
	\end{align*}
\end{bsp}
\section{Ordnung und Index}

Sei $G$ eine Gruppe und $g \in G$.

\begin{defin}
	\begin{enumerate}
		\item $\# G = \card{G} \in \nat \cup \{\infty\}$, die Ordnung von $G$.
		\item $\ord(g) = \# \erz{g}$, die Ordnung von $g$.
	\end{enumerate}
\end{defin}
\begin{bsp}
	$\# S_n = n! \quad , \quad \# A_n = \simfrac{2}*n! \quad (n\geq 2)$ \\
	$\# \rest{n} = n$
\end{bsp}

\begin{lemma}
	Für $X \subseteq G$ ist
	\begin{align*}
		\erz{X} = \{ g_1^{\epsilon_1}  * \cdots * g_r^{\epsilon_r} : r \in \nat_0, g_1,\dots,g_r \in X, \epsilon_1 ,\dots, \epsilon_r \in \{1,-1\} \}
	\end{align*}
\end{lemma}
\begin{proof}
	klar, da rechte Seite Untergruppe ist, die $X$ enthält, und jede solche enthält alle  Ausdrücke der Form $g_1^{\epsilon_1}, \cdots g_r^{\epsilon_r}$.
\end{proof}

\begin{satz}
	\begin{enumerate}
		\item Ist $\ord(g)= \infty$, so ist $\erz{g} = \{ \dots , g^{-2}, g^{-1}, 1, g, g^2, \dots \}$.
		\item Ist $\ord(g)=n < \infty$, so ist $\erz{g}=\{1,g,g^2, \dots , g^{n-1} \}$.
		\item Es ist $\ord(g) = \inf\{ k \in \nat : g^k=1 \}$.
	\end{enumerate}
\end{satz}
\begin{proof}
	Nach 2.3 ist $\erz{g}= \{ g^k : k \in \integer \}$. Sei $m=\inf\{k \in \nat : g^k = 1 \}$.
	\begin{itemize}
		\item $| \{ g^k : 0 \leq k < m \} | = m$: Sind $g^a=g^b$ mit $0 \leq a < b < m$, so ist $g^{b-a}=1$, aber $0 < b-a < m$ im Widerspruch zur Minimalität von m.
		\item $m=\infty$: $\follows \ord(g)=\infty \quad$ \checkmark
		\item $m<\infty$: $\follows \erz{g} = \{ g^k : 0 \leq k < m \}:$ Die Inklusion $\{ g^k :0 \leq k < m \} \subseteq \erz{g}$ ist klar. Für die andere Inklusion schreibe $k \in \integer$ als $k=g*m+r$ mit $q,r \in \integer, 0 \leq r < m$. \\ $\follows g^k =g^{q*m+r} = (g^m)^q*g^1 =1^q*g^r =g^r \in \{ 1,g,\dots,g^{m-1} \}$ \\
		$\follows \erz{g} \subseteq  \{ g^k : 0\leq k < m \}$
	\end{itemize}
\end{proof}

\begin{bsp}
	Sei $\sigma \in S_n$ ein $k$-Zykel, so ist $\ord(\sigma)=k$. \\ \textcolor{gray}{(Man muss genau k-mal tauschen bis alle Elemente wieder an ihrem Platz sind)} \\
	Für $\bar{1} \in \rest{n}$ ist $\ord(\bar{1})=n$. \\ \textcolor{gray}{($n*\bar{1}=\bar{n}=\bar{0} \in \rest{n}$)}
\end{bsp}

\begin{defin}
	Seien $A,B \subseteq G, H \leq G$.
	\begin{itemize}
		\item $AB := A*B := \{ a*b : a \in A, b \in B \}$, das Komplexprodukt von $A$ und $B$
		\item $gH := \{g\}*H = \{ g*h : h \in H\}$, die Linksnebenklasse von $H$ bezüglich $g$ \\
		$Hg := H * \{g\} = \{ h*g : h \in H\}$, die Rechtsnebenklasse von $H$ bezüglich $g$
		\item $G/H := \{ gH : g \in G \} \quad$ \textcolor{gray}{(Menge der Linksnebenklassen)} \\
		$H \backslash G := \{ Hg : g \in G \} \quad$ \textcolor{gray}{(Menge der Rechtsnebenklassen)} 
	\end{itemize}
\end{defin}

\begin{bsp}
	Für $h \in H$ ist $hH = H = Hh$ (vgl. 1.4).
\end{bsp}

\begin{lemma}
	Seien $H \leq G, \, g,g' \in G$.
	\begin{enumerate}
		\item $gH = g'H \equivalent g' = gh \in G$ für ein $h \in H$. \\
		$Hg = Hg' \equivalent g' = hg \in G$ für eine $h \in H$
		\item Es ist $gH = g'H$ oder $gH \cup g'H = \emptyset$ und $Hg = Hg'$ oder $Hg \cup Hg' = \emptyset$.
		\item Durch $gH \mapsto Hg^{-1}$ wird eine wohldefinierte Bijektion $G/H \to H \backslash G$ gegeben.
	\end{enumerate}
\end{lemma}
\begin{proof}
	Seien $H \leq G, \, g,g' \in G$.
	\begin{enumerate}
		\item $(\Rightarrow): gH = g'H \follows g' = g' * 1 \in g'H=gH \follows \exists h \in H: g' = gh$. \\
		$(\Leftarrow): g' = gh \follows g'H = ghH \overset{2.7}{=} gH$
		\item Ist $gH \cap g'H \neq \emptyset$, so existieren $h,h' \in H$ mit $gh =g'h'$. $\follows gH=ghH=g'h'H=g'H$
		\item Wohldefiniertheit: $gH=g'H \enspace \overset{\text{a)}}{\Rightarrow} \enspace g'h=gh$ mit $h \in H \enspace \Rightarrow \enspace H(g')^{-1} =Hh^{-1}g^{-1} = Hg^{-1}$ \\
		Bijektivität: klar, da Umkehrabbildung $Hg \mapsto g^{-1}H$
	\end{enumerate}
\end{proof}

\begin{bsppure}
	Betrachte $S_3$ als kleinste nicht-abelsche Gruppe.
\end{bsppure}

\begin{defin}
	Für $H \leq G$ ist
	\begin{align*}
		(G:H) := \# G/H \overset{2.8c}{=} \# h \backslash G \in \nat \cup {\infty}
	\end{align*}
	der Index von $H$ in $G$.
\end{defin}

\begin{bsp}
	$(S_n : A_n) = 2 \qquad (n \geq 2)$ \\
	$(\integer : n\integer) = n$
\end{bsp}

\begin{satz}
	Der Index ist multiplikativ: Sind $K \leq H \leq G$, so ist 
	\begin{align*}
		(G:K) = (G:H) * (H:K)
	\end{align*}
\end{satz}
\begin{proof}
	Nach 2.8b bilden die Nebenklassen von $H$ eine Partition von $G$, d.h. es gibt eine Familie $(g_i)_{i \in I}$ in $G$ mit $G = \bigcup \limits_{i \in I}{g_i H}$ ($G= \bigcup \limits_{i \in I}{g_i H}$ und $g_i H, i \in I$ sind paarweise disjunkt). \\
	Analog ist $H = \bigcup \limits_{j \in J}{h_j K}$ mit $h_j \in H$. Dann gilt: 
	\begin{align*}
		H = \bigcup\limits_{j \in J}{h_j K} &\overset{1.4}{\follows} gH = \bigcup\limits_{j \in J}{g \, h_j \, K} \quad \text{ für jedes } g \in G \\
		&\follows G = \bigcup \limits_{i \in I}{g_i H} = \bigcup \limits_{i \in I}{\bigcup\limits_{j \in J}{g_i h_j K}} = \bigcup \limits_{(i,j) \in I \times J}{g_i h_j K}
	\end{align*}
	Somit ist $(G:K) = \card{I \times J} = \card{I} * \card{J} = (G:H) * (H:K)$.
\end{proof}

\begin{kor} [Satz von Lagrange (wichtigster Satz der Vorlesung)]
	Ist $G$ endlich und $H \leq G$, so ist
	\begin{align*}
		\# G = \# H * (G:H)
	\end{align*}
	Insbesondere gilt $\# H \teilt \# G$ und $(G:H) \teilt \# G$.
\end{kor}
\begin{proof}
	$\# G = (G:1) \overset{2.11}{=} (G:H) * (H:1) = (G:H) * \# H$
\end{proof}

\begin{kor}[Kleiner Satz von Fermat]
	Ist $G$ endlich und $n = \# H$, so ist $g^n=1$ für jedes $g \in G$.
\end{kor}
\begin{proof}
	Nach 2.12 gilt $\ord(g) = \# \erz{g} \teilt \# G = n$. Nach 2.4 ist $g^{\ord(g)}=1$, somit auch
	\begin{align*}
		g^n=(\underbrace{g^{\ord(g)}}_{=1})^{\frac{n}{\ord(g)}} =1
	\end{align*}
\end{proof}
\begin{bem}
	Nach 2.12 ist die Ordnung jeder Untergruppe von $G$ ein Teiler der Gruppenordnung $\# G$. Umgekehrt gibt es im Allgemeinen aber nicht zu jedem Teiler $d$ von $\# G$ eine Untergruppe $H$ von $G$ mit $\# H =d$.
\end{bem}
\input{geo_3_normalteiler-quotientengruppe}
\section{Abelsche Gruppen}
%
Sei $G$ eine Gruppe.
%
\begin{defin}
	$G$ ist \begriff{zyklisch} $\equivalent G = \erz{g}$ für ein $g \in G$.
\end{defin}
%
\begin{bsp}
	\begin{enumerate}
		\item $\integer$ ist zyklisch.
		\item $\rest{n}$ ist zyklisch der Ordnung $n$.
		\item $C_n = \erz{(1 \, 2 \, \cdots \, n)} \leq S_n$ ist zyklisch der Ordnung $n$.
		\item Ist $\#G = p$ eine Primzahl, so ist $G$ zyklisch (vgl. Ü6).
	\end{enumerate}
\end{bsp}
%
\begin{lemma} [!]
	Die Untergruppen von $(\integer, +)$ sind genau die $\erz{k} = k\integer$ mit $k \in \nat_0$ und für $k_1, \dots, k_r \in \integer$ ist $\erz{k_1, \dots , k_r} = \erz{k}$ mit $\ggT(k_1, \dots , k_r) = k$.
\end{lemma}
\begin{proof}
	Jede Untergruppe von $\integer$ ist ein Ideal von $(\integer,+,*)$ und $\integer$ ist Hauptidealring.
\end{proof}
\begin{proof}
	Sei $H \leq \integer$. Setze $k = \min (H \cap N)$, o.E. sei $H \neq \menge{0}$. Offensichtlich gilt $\erz{k} \subseteq H$.
	\begin{itemize}
		\item $n \in H \follows n = q*k + r$ mit $q,r \in \integer, 0 \leq r < k \follows r = n - \underbrace{g*k}_{k+\cdots+k}$ \\
		$\follows r = 0 \text{wegen der Minimalität von } k$ \\
		$\follows n = q*k$, d.h. $n \in \erz{k}$
		\item $k = \ggT(k_1, \dots , k_r)$: \\
		$k_i \in \erz{k} \follows k \teilt k_i \enspace \forall i$ \\
		$k \in \erz{k_1, \dots , k_r} \follows k = n_1 k_1 + \cdots + n_r k_r$ mit $n_i \in \integer$ \\
		$d \teilt k_i \enspace \forall i \follows d \teilt k$
	\end{itemize}
\end{proof}
%
\begin{satz}
	Sei $G = \erz{g}$ zyklisch. Dann ist $G$ abelsch und $G \isomorph (\integer,+)$ oder $G \isomorph (\rest{n}, +)$ mit $n = \# G < \infty$.
\end{satz}
\begin{proof}
	Betrachte
	\begin{align*}
		\phi \colon \begin{cases} \integer \to G \\ k \mapsto g^k \end{cases}
	\end{align*}
	$\phi$ ist Homomorphismus und surjektiv, da $G = \erz{g}$. Nach 3.9. ist $G = \bild(\phi) \isomorph \integer / \Ker(\phi)$. Nach 4.3 ist $\Ker(\phi)  = \erz{n}$ für ein $n \in \nat_0$. Ist $n=0$, so ist $\Ker(\phi) = \menge{0}$ und $G \isomorph \integer$. Ist $n > 0$, so ist $G \isomorph \rest{n}$ und $n = \# \rest{n} = \# G$.
\end{proof}
%
\begin{satz}
	Sei $(G,+) = \erz{g}$ zyklisch der Ordnung $n \in \nat$.
	\begin{enumerate}
		\item Zu jedem $d \in \nat$ mit $d \teilt n$ hat $G$ genau eine Unterguppe der Ordnung $d$, nämlich \\
		$U_d := \erz{\frac{n}{d} \enspace g}$. \textcolor{gray}{Damit ist jede Untergruppe einer zyklischen Gruppe wieder zyklisch.}
		\item Für $d \teilt h$ und $d' \teilt h'$ ist $U_d \subseteq U_{d'} \follows d \teilt d'$.
		\item Für $k_1 , \dots k_r \in \integer$ ist $\erz{k_1g, \dots , k_r g} = \erz{eg} = U_{\frac{n}{e}}$ mit $e = \ggT(k_1, \dots , k_r , n)$.
		\item Für $k \in \integer$ ist $\ord(kg) = \frac{n}{\ggT(k,n)}$.
	\end{enumerate}
\end{satz}
\begin{proof}
	Betrachte wieder $\abb{\phi}{\integer}{G}, k \mapsto kg$.
	\begin{enumerate}
		\item Nach 3.7 und 4.3 liefert $\phi$ eine Bijektion $\menge{e \in \integer : n\integer \leq e\integer} \to \menge{H \leq G}$ und \\
		$n\integer \leq e\integer \equivalent e \teilt n$. \\
		Ist $H = \phi(e\integer) = \erz{eg}$, so ist $H \isomorph e\integer / n\integer$ \textcolor{gray}{($n\integer = \ker(\phi)$)}, also
		\begin{align*}
			n = (\integer : n \integer) = (\integer : e\integer) (e\integer : n \integer) = e * \# H
		\end{align*}
		\item $U_d \subseteq U_{d'} \equivalent \erz{\frac{n}{d} \enspace g} \subseteq \erz{\frac{n}{d'} \enspace g} \equivalent \frac{n}{d} \integer \leq \frac{n}{d'} \integer \follows \frac{n}{d'} \teilt \frac{n}{d} \equivalent d \teilt d'$
		\item Mit $H = \erz{k_1 , \dots , k_r} \leq \integer$ ist $n\integer \leq H$, $\phi(H) = \erz{k_1 g , \dots , k_r g}$ \textcolor{gray}{($n \in \ker(\phi)$)}. \\
		Nach 4.3 ist $H = \erz{e}$ mit $e = \ggT(k_1 , \dots , k_r , n)$ und somit $\erz{k_1 g , \dots k_r g} = \phi(e\integer) = U_{\frac{n}{e}}$.
		\item $\ord(kg) = \# \erz{kg} \overset{(c)}{=} U_{\frac{n}{e}}$ mit $e = \ggT(k,n)$. \textcolor{gray}{(Fall (c) mit $r=1$)}
	\end{enumerate}
\end{proof}
%
\begin{lemma}
	Seien $a,b \in G$. Kommutieren $a$ und $b$ und sind $\ord(a), \ord(b)$ teilerfremd, so ist \\
	$\ord(a*b) = \ord(a) * \ord(b)$.
\end{lemma}
\begin{proof}
	Nach 2.12 ist $\erz{a} \cap \erz{b} = \menge{1}$. Ist $(ab)^k = a^k * b^k$, so ist $a^k = b^{-k} \in \erz{a} \cap \erz{b} = 1$, also $a^k = b^k = 1$. Somit ist $(ab)^k = 1 \equivalent a^k = 1$ und $b^k = 1$, und damit
	\begin{align*}
		\ord(ab) = \kgV(\ord(a),\ord(b)) = \ord(a) * \ord(b)
	\end{align*}
\end{proof}
%
\begin{kor}
	Ist $G$ abelsch und sind $a,b \in G$ mit $\ord(a) = < \infty, \ord(b) = n < \infty$, so existiert ein $c \in G$ mit $\ord(c) = \kgV(\ord(a), \ord(b))$.
\end{kor}
\begin{proof}
	Schreibe $m = m_0 * m'$ und $n = n_0 * n'$ mit $m_0 * n_0 = \kgV(m,n)$ und $\ggT(m_0,n_0) = 1$. \\
	$\follows \ord(a^{m'}) = m_0 \enspace , \enspace \ord(b^{n'}) = n_0$ \\
	$\follows \ord(a^{m'}b^{n'}) \overset{4.6}{=} m_0*n_0 = \kgV(m,n)$ \\
	Dann ist $c:= a^{m'}b^{n'}$.
\end{proof}
%
\begin{thm}[Struktursatz für endlich erzeugte abelsche Gruppen]
	Jede endlich erzeugte abelsche Gruppe $G$ ist eine direkte Summe zyklischer Gruppen
	\begin{align*}
		G^r \isomorph \integer^r \oplus \bigoplus\limits_{i=1}^{k}{\rest{d_i}}
	\end{align*}
	mit eindeutig bestimmten $d_i, \dots, d_k > 1$, die $d_i \teilt d_{i+1}$ für alle $i$ erfüllen.
\end{thm}
\begin{proof}
	Die Existenz folgt aus LAAG VIII.6.14 (Hauptsatz über endlich erzeugte Moduln über Hauptidealringen). \\
	Eindeutigkeit: Für $d \in \nat$ ist $\# G/dG = \# (\rest{d})^r \oplus \bigoplus_{i=1}^{k}{(\rest{d_i})/d*\rest{d_i}} \overset{4.5(d)}{=} d^r * \prod_{i=1}^{k}{\frac{d_i}{\ggT(d,d_i)}}$. Daraus kann man nun $r,k,d_1, \dots, d_k$ erhalten, z.B. für $p$ prim $p \teilt nicht d_i \enspace \forall i$ ist $\# G/pG = p^r * \prod_{i=1}^{k}{d_i}$.
\end{proof}
%
\begin{lemma}
	Sei $G = (G,+) = \erz{g}$ zyklisch der Ordnung $n \in  \nat$. Die Endomorphismen von $G$ sind genau die 
	\begin{align*}
		\phi_{\quer{k}}: \begin{cases}
		G \to G \\
		x \mapsto kx
		\end{cases}
	\end{align*}
	für $\quer{k} = k +n\integer \in \rest{n}$. Dabei ist $\phi_{\quer{l}} \circ \phi_{\quer{k}} = \phi_{\quer{kl}}$ für $\quer{k}, \quer{l} \in \rest{n}$.
\end{lemma}
\begin{proof}
	Zu zeigen sind eine Reihe von Aussagen.
	\begin{itemize}
		\item $\phi_{\quer{k}}$ wohldefiniert: $\quer{k_1} = \quer{k_2} \follows k_2 = k_1 + an$ mit $a \in \integer$. Dann ist auch $k_2 x = k_1 x + a*nx = k_1 x \enspace \forall x \in G$.
		\item $\phi_{\quer{k}} \in \Hom(G,G)$: klar, da $G$ abelsch.
		\item $\quer{k} = \quer{l} \equivalent \phi_{\quer{k}} = \phi_{\quer{l}}$: $\quad$ \textcolor{gray}{(zeige $\Leftarrow$; $\Rightarrow$ ist Wohldefiniertheit)} \\
		$\phi_{\quer{k}} = \phi_{\quer{l}} \follows \phi_{\quer{k}} (g) = \phi_{\quer{l}} (g) \follows (k-l) g = 0 \overset{\ord(g)=n}{\follows} n \teilt k-l \follows \quer{k} = \quer{l}$
		\item $\phi \in \Hom(G,G)$: $\follows \phi = \phi_{\quer{k}}$ für ein $k \in \integer$ ; $\phi(g) = k*g$ für ein $k \in \integer \follows \phi = \phi_{\quer{k}}$
		\item $\phi_{\quer{l}} \circ \phi_{\quer{k}} = \phi_{\quer{kl}}$: $\quad l*(k*x) = (l*k) * x$ \checkmark
	\end{itemize}
\end{proof}
%
\begin{satz}
	Ist $G$ zyklisch von Ordnung $n \in \nat$, so ist $\Aut(G) \isomorph (\rest{n})^{\times}$. \textcolor{gray}{(multiplikativ)}
\end{satz}
\begin{proof}
	$\Aut(G) \subseteq \Hom(G,G) = \menge{\phi_{\quer{k}} : \quer{k} \in \rest{n}}$. \\
	\begin{align*}
		\phi_{\quer{k}} \in \Aut(G) &\equivalent \exists \quer{l} \in \rest{n}: \phi_{\quer{l}} \circ \phi_{\quer{k}} = \phi_{\quer{1}} \\
		&\equivalent \exists \quer{l} \in \rest{n}: \quer{l} * \quer{k} = 1 \\
		&\equivalent \quer{k} \in \einheit{(\rest{n})}
	\end{align*}
	und die Abbildung $\einheit{(\rest{n})} \to \Aut(G)$ mit $\quer{k} \mapsto \phi_{\quer{k}}$ ist ein Isomorphismus. \textcolor{gray}{Offensichtlich ist diese ein Homomorphismus und die Bijektivität folgt aus der Tatsache, dass jeder Endomorphismus genau diese Gestalt $\phi_{\quer{k}}$ hat.}
\end{proof}
%
\begin{defin}
	Die Abbildung $\abb{\Phi}{\nat}{\nat}$ gegeben durch
	\begin{align*}
		\Phi(n) = \# \einheit{(\rest{n})}
	\end{align*}
	ist die \begriff{Euler'sche Phi-Funktion}.
\end{defin}
%
\begin{bsp}
	Ist $p$ prim, so ist $\Phi(p) = p-1$, da $\rest{p}$ ein Körper ist.
\end{bsp}
\input{geo_5_produkte}

\end{document}

