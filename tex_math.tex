%
% ================================ 
% | MATH SETTINGS                |
% ================================

%============= BUCHSTABEN =============
\renewcommand{\epsilon}{\varepsilon}
\renewcommand{\phi}{\varphi}

%============= LOGIK =============
\newcommand{\follows}{\quad \Rightarrow \quad}
\newcommand{\equivalent}{\quad \Leftrightarrow \quad}

%============= MENGEN =============
\newcommand{\menge}[1]{\{ #1 \}}
\newcommand{\card}[1]{\abs{#1}}

\newcommand{\nat}{\mathbb{N}}				% natürliche Zahlen
\newcommand{\integer}{\mathbb{Z}}			% ganze Zahlen
\newcommand{\ratio}{\mathbb{Q}}
\newcommand{\real}{\mathbb{R}}				% reelle Zahlen
\newcommand{\complex}{\mathbb{C}}			% komplexe Zahlen
\newcommand{\korper}{\mathbb{K}}			% Körper IR oder IC

\newcommand{\einheit}[1]{{#1}^{\times}}
\newcommand{\ringx}{\einheit{R}}			% Einheiten Ring
\newcommand{\korperx}{\einheit{K}}			% Einheiten Körper

\newcommand{\rest}[1]{\integer / #1 \integer} % Restklassen Z/( )Z
% add Polynomring
% add Endlicher Körper

\newcommand{\ohneNull}{\backslash \menge{0}}% Menge ohne Null \{0}

\DeclareMathOperator{\inn}{int} % Set of inner points
\DeclareMathOperator{\ext}{ext} % Set of outer points
\DeclareMathOperator{\cl}{cl} 	% Abschluss
\DeclareMathOperator{\diam}{diam}
\newcommand{\rand}{\partial}

%============= FUNKTIONEN =============
\newcommand{\abb}[3]{#1 \colon #2 \to #3}	  % Abbildungen mit Namen, einzeilig
\newcommand{\bigabb}[5]{#1 \colon \left\lbrace%	Abbildungen mit Namen, zweizeilig
	\begin{array}{ccl}%
		#2 & \to & #3 \\%
		#4 & \mapsto & #5%
	\end{array}%
	\right.}
%
\newcommand{\bigabbnoname}[4]{\left\lbrace%		Abbildungen ohne Namen, zweizeilig
	\begin{array}{ccl}%
		#1 & \to & #2 \\%
		#3 & \mapsto & #4%
	\end{array}%
	\right.}

\DeclareMathOperator{\id}{id}
\DeclareMathOperator{\graph}{graph}
\DeclareMathOperator{\grad}{grad}
\newcommand{\quer}[1]{\overline{#1}}

%============= ZAHLENTHEORIE =============
\newcommand{\teilt}{\mid}
\newcommand{\teiltnicht}{\nmid}
\newcommand{\normalteiler}{\trianglelefteq}
\newcommand{\nichtnormal}{\ntrianglelefteq}
\DeclareMathOperator{\ggT}{ggT}
\DeclareMathOperator{\kgV}{kgV}

\newcommand{\simfrac}[1]{\frac{1}{#1}}

%============= VEKTORRÄUME =============
\newcommand{\basisB}{\mathcal{B}}
\newcommand{\darstMat}[2]{M_{\mathcal{#1}} (#2)}
\newcommand{\MBf}{\darstMat{\basisB}{f}}
\newcommand{\one}{\mathbbm{1}}

\newcommand{\isomorph}{\cong}

\newcommand{\erz}[1]{\langle #1 \rangle}
\newcommand{\scal}[2]{\langle #1\, ,\, #2 \rangle}

\DeclareMathOperator{\Span}{span}

\DeclareMathOperator{\Hom}{Hom}
\DeclareMathOperator{\End}{End}
\DeclareMathOperator{\Aut}{Aut}
\DeclareMathOperator{\Ker}{Ker}
\DeclareMathOperator{\bild}{Im}

\DeclareMathOperator{\Mat}{Mat}
\DeclareMathOperator{\diag}{diag}
\DeclareMathOperator{\GL}{GL}
\DeclareMathOperator{\SL}{SL}
\DeclareMathOperator{\rang}{rang}
\DeclareMathOperator{\tr}{tr}
\newcommand{\transpose}[1]{\left( #1 \right)^{\top}}
\newcommand{\trans}[1]{#1^{\top}}
\DeclareMathOperator{\cond}{cond}

\DeclareMathOperator{\sgn}{sgn}
\DeclareMathOperator{\Sym}{Sym}

\DeclareMathOperator{\Eig}{Eig}

%============= GRUPPEN =============
\DeclareMathOperator{\ord}{ord}		% Ordnung eines Elements
\DeclareMathOperator{\Inn}{Inn}		% Abbildung Inn
\DeclareMathOperator{\zentrum}{Z}	% Zentrum einer Gruppe
\DeclareMathOperator{\fix}{Fix}		% Fixpunkte einer Wirkung
\DeclareMathOperator{\stab}{Stab}	% Stabilisator
\DeclareMathOperator{\Syl}{Syl}		% Menge der p-Sylowgruppen
\DeclareMathOperator{\N}{N}          % Normalisator
\DeclareMathOperator{\C}{C}           % Zentralisator
\DeclareMathOperator{\Z}{Z}           % Zentrum
\DeclareMathOperator{\typ}{Typ}  % Typ einer Permutation aus S_n

%============= RINGE =============

%============= KÖRPER =============
\DeclareMathOperator{\GF}{GF}


%============= METRISCHE RÄUME =============
\newcommand{\abs}[1]{\left| #1 \right|}
\newcommand{\norm}[1]{\left\lVert #1 \right\rVert}

%============= DIFFERENZIERUNG =============
\newcommand{\partdiff}[1]{\frac{\partial}{\partial #1 }}
\newcommand{\ableitung}[1]{\frac{\mathrm{d}}{\mathrm{d} #1 }}
\newcommand{\diff}[1]{\enspace \mathrm{d}#1}
\newcommand{\dx}{\diff{x}}
\newcommand{\dy}{\diff{y}}
\newcommand{\da}{\diff{a}}

\DeclareMathOperator{\Lin}{L}		% Menge der lineare Abb.
\DeclareMathOperator{\cdiff}{C}		% Menge der stetig-diffbaren Abb.

%============= INTEGRATION =============
\DeclareMathOperator{\tangentialraum}{T}
\DeclareMathOperator{\normalenraum}{N}

\DeclareMathOperator{\divergenz}{div}
\renewcommand{\div}{\divergenz}
\DeclareMathOperator{\rot}{rot}



