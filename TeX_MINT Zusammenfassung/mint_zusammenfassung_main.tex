\documentclass[a4paper, 11pt, ngerman]{scrartcl}
\KOMAoptions{%
	parskip=half,%
	fontsize=11pt}

% === INPUT ===================================================

\usepackage{mathpazo}
\usepackage{eulervm}
\input{../tex_preset/tex_packages}
%============= BUCHSTABEN =============
\renewcommand{\epsilon}{\varepsilon}
\renewcommand{\phi}{\varphi}

%============= LOGIK =============
\newcommand{\follows}{\quad \Rightarrow \quad}
\newcommand{\equivalent}{\quad \Leftrightarrow \quad}

%============= MENGEN =============
\newcommand{\nat}{\mathbb{N}}				% natürliche Zahlen
\newcommand{\integer}{\mathbb{Z}}			% ganze Zahlen
\newcommand{\real}{\mathbb{R}}				% reelle Zahlen
\newcommand{\complex}{\mathbb{C}}			% komplexe Zahlen
\newcommand{\korper}{\mathbb{K}}			% Körper IR oder IC
\newcommand{\ringx}{R^{\times}}				% Einheiten Ring
\newcommand{\korperx}{K^{\times}}			% Einheiten Körper
\newcommand{\rest}[1]{\integer / #1 \integer} % Restklassen Z/( )Z
\newcommand{\ohneNull}{\backslash \menge{0}}% Menge ohne Null \{0}
\newcommand{\einheit}[1]{{#1}^{\times}}

\newcommand{\menge}[1]{\{ #1 \}}
\newcommand{\card}[1]{\abs{#1}}
\newcommand{\st}{\,|\,}
\DeclareMathOperator{\inn}{int} % Set of inner points
\DeclareMathOperator{\ext}{ext} % Set of outer points
\DeclareMathOperator{\cl}{cl} 	% Abschluss
\DeclareMathOperator{\diam}{diam}

%============= FUNKTIONEN =============
\newcommand{\abb}[3]{#1 \colon #2 \to #3}
\DeclareMathOperator{\id}{id}
\DeclareMathOperator{\graph}{graph}
\DeclareMathOperator{\grad}{grad}
\newcommand{\quer}[1]{\bar{#1}}

%============= ZAHLENTHEORIE =============
\newcommand{\teilt}{\mid}
\newcommand{\normalteiler}{\trianglelefteq}
\newcommand{\nichtnormal}{\ntrianglelefteq}
\DeclareMathOperator{\ggT}{ggT}
\DeclareMathOperator{\kgV}{kgV}

\newcommand{\simfrac}[1]{\frac{1}{#1}}

%============= VEKTORRÄUME =============
\newcommand{\basisB}{\mathcal{B}}
\newcommand{\darstMat}[2]{M_{\mathcal{#1}} (#2)}
\newcommand{\MBf}{\darstMat{\basisB}{f}}
\newcommand{\one}{\mathbbm{1}}

\newcommand{\isomorph}{\cong}

\newcommand{\erz}[1]{\langle #1 \rangle}
\newcommand{\scal}[2]{\langle #1\, ,\, #2 \rangle}

\DeclareMathOperator{\Span}{span}

\DeclareMathOperator{\Hom}{Hom}
\DeclareMathOperator{\End}{End}
\DeclareMathOperator{\Aut}{Aut}
\DeclareMathOperator{\Ker}{Ker}
\DeclareMathOperator{\bild}{Im}

\DeclareMathOperator{\Mat}{Mat}
\DeclareMathOperator{\diag}{diag}
\DeclareMathOperator{\GL}{GL}
\DeclareMathOperator{\SL}{SL}
\DeclareMathOperator{\rang}{rang}
\DeclareMathOperator{\tr}{tr}
\newcommand{\transpose}[1]{\left( #1 \right)^{\top}}

\DeclareMathOperator{\sgn}{sgn}
\DeclareMathOperator{\Sym}{Sym}

\DeclareMathOperator{\Eig}{Eig}

%============= GRUPPEN =============
\DeclareMathOperator{\ord}{ord}
\DeclareMathOperator{\Inn}{Inn}
\DeclareMathOperator{\zentrum}{Z}
\DeclareMathOperator{\fix}{Fix}
\DeclareMathOperator{\stab}{Stab}

%============= RINGE =============

%============= METRISCHE RÄUME =============
\newcommand{\abs}[1]{\left| #1 \right|}
\newcommand{\norm}[1]{\left\lVert #1 \right\rVert}

%============= DIFFERENZIERUNG =============
\newcommand{\partdiff}[1]{\frac{\partial}{\partial #1 }}
\newcommand{\diff}[1]{\enspace \mathrm{d}#1}
\newcommand{\dx}{\diff{x}}
\newcommand{\dy}{\diff{y}}
\newcommand{\da}{\diff{a}}

\DeclareMathOperator{\Lin}{L}		% Menge der lineare Abb.
\DeclareMathOperator{\cdiff}{C}		% Menge der stetig-diffbaren Abb.

%============= INTEGRATION =============
\DeclareMathOperator{\tangentialraum}{T}
\DeclareMathOperator{\normalenraum}{N}
\DeclareMathOperator{\divergenz}{div}
\renewcommand{\div}{\divergenz}
\DeclareMathOperator{\rot}{rot}
\newcommand{\rand}{\partial}

\newcommand{\bigabb}[5]{#1 \colon \left\lbrace%
\begin{array}{ccl}%
	#2 & \to & #3 \\%
	#4 & \mapsto & #5%
\end{array}%
\right.}
%
\newcommand{\bigabbnoname}[4]{\left\lbrace%
\begin{array}{ccl}%
	#1 & \to & #2 \\%
	#3 & \mapsto & #4%
\end{array}%
\right.}

\input{../tex_preset/tex_theoreme}
\input{mint_localsettings}

\begin{document}

\setlength\abovedisplayshortskip{3pt}
\setlength\belowdisplayshortskip{3pt}
\setlength\abovedisplayskip{3pt}
\setlength\belowdisplayskip{3pt}

\section{$\sigma$-Algebren und Maße}
\begin{defin}
	Eine $\sigma$-Algebra über einer Grundmenge $X\neq \emptyset$ ist eine Familie $\mathcal{A} \subseteq \mathcal{P}(X)$ von Mengen mit
	\begin{itemize}[noitemsep]
		\item[($\Sigma_1$)] $X \in \mathcal{A}$
		\item[($\Sigma_2$)] $A \in \mathcal{A} \follows A^\complement = X \backslash A \in \mathcal{A}$
		\item[($\Sigma_1$)] $(A_n)_{n \in \nat} \subseteq \mathcal{A} \follows \bigcup_{n \in \nat}{A_n} \in \mathcal{A}$
	\end{itemize}
\end{defin}
\begin{satz}
	Sei $\mathcal{A}$ eine $\sigma$-Algebra über $X$.
	\begin{enumerate}
		\item $\emptyset \in \mathcal{A}$
		\item $A,B \in \mathcal{A} \follows A \cup B \in \mathcal{A}$ mittels ($\Sigma_3$).
		\item $(A_n)_{n \in \nat} \subseteq \mathcal{A} \follows \bigcap_{n \in \nat}{A_n} \in \mathcal{A}$ (Schnitt = Vereinigung von Komplementen, de Morgan)
		\item $A,B \in \mathcal{A} \follows A \cap B \in \mathcal{A}$
		\item $A,B \in \mathcal{A} \follows A \backslash B \in \mathcal{A}$
	\end{enumerate}
\end{satz}

\begin{bsp}
	Die üblichen Beispiele $\mathcal{P}(X), \menge{\emptyset, X}, \dots$ sind klar.
	\begin{itemize}
		\item \textbf{Spur-$\sigma$-Algebra.} $E \subseteq X$ beliebig, $\mathcal{A}$ $\sigma$-Algebra in $X$. $\follows \mathcal{A}_{E} := \menge{E \cap A : A \in \mathcal{A}}$ ist $\sigma$-Algebra.
		\item \textbf{Urbild-$\sigma$-Algebra.} $\abb{f}{X}{X'}$ eine Abbildung, $X,X'$ Mengen, $\mathcal{A}'$ sei $\sigma$-Algebra in $X'$. Dann ist $\mathcal{A} := \menge{f^{-1}(A') : A' \in \mathcal{A}'}$ eine $\sigma$-Algebra.
	\end{itemize}
\end{bsp}

\begin{defin}
	Die von $\mathcal{O} = \mathcal{O}(\real^d) := \menge{U \subseteq \real^d : U \text{ offen}}$ erzeugte $\sigma$-Algebra in $\real^d$ heißt \begriff{Borel-$\sigma$-Algebra} $\mathcal{B}(\real^d)$.
\end{defin}

\noindent\rule{4cm}{0.4pt}

\begin{defin}
	Ein Maß ist eine Abbildung $\abb{\mu}{\mathcal{A}}{[0,\infty]}$ mit
	\begin{itemize}[noitemsep]
		\item[($M_0$)] $\mathcal{A}$ ist $\sigma$-Algebra.
		\item[($M_1$)] $\mu(\emptyset) = 0$.
		\item[($M_2$)] $(A_n)_{n \in \nat}$ paarweise disjunkt. $\follows \mu\left( \biguplus_{n \in \nat}{A_n} \right) = \sum_{n \in \nat}{\mu(A_n)} \in \mathcal{A}$
	\end{itemize}
	Ein Maß $\mu$ heißt \begriff{endlich}, falls $\mu(X) < \infty$ und \begriff{$\sigma$-endlich}, falls
	\begin{align*}
	\exists (A_n)_{n \in \nat} \subseteq \mathcal{A}, A_n \uparrow X : \mu(A_n) < \infty \text{ für alle } n \in \nat
	\end{align*}
\end{defin}

\begin{thm}[Eindeutigkeitssatz]
	Sei $(X,\mathcal{A})$ ein beliebiger Messraum, $\mu, \nu$ zwei Maße und $\mathcal{A} = \sigma(\mathcal{G})$. Weiter sei $\mathcal{G}$ $\cap$-stabil und $\exists (G_n)_{n \in \nat} \subseteq \mathcal{G}, G_n \uparrow X$ : $\mu(G_n), \nu(G_n) < \infty$. Dann gilt
	\begin{align*}
	\forall G \in \mathcal{G}: \mu(G) = \nu(G) &\follows \forall A \in \sigma(\mathcal{G}) : \mu(A) = \nu(A) \\
	\forall G \in \mathcal{G} : \mu |_G = \nu |_G &\follows \mu = \nu
	\end{align*}
\end{thm}
\begin{proof}
	Beweisidee: Wunschkonzert \\
	Zeige, dass $\mathcal{D}_n := \menge{A \in \mathcal{A} : \mu(G_n \cap A) = \nu(G_n \cap A)}$ für alle $n \in \nat$ ein Dynkin-System ist ($X \in \mathcal{D}$, $D^\complement \in \mathcal{D}$, $\biguplus_{n \in \nat}{D_n} \in \mathcal{D}$). \\
	Verwende die Eigenschaft $\sigma(\mathcal{G}) = \delta(\mathcal{G})$ für $\cap$-stabile $\mathcal{G}$ um zu zeigen, dass $\mathcal{A} = \mathcal{D}_j$ \\
	Nutze Maßstetigkeit um $\mu(A \cap G_n) = \nu(A \cap G_n) \leadsto \mu(A) = \nu(A)$.
\end{proof}

\begin{thm}[Fortsetzungssatz nach Carathéodory]
	Sei $\mathcal{S}$ ein Halbring über $X$ und $\abb{\mu}{\mathcal{S}}{[0,\infty]}$ ein Prämaß, d.h. $\mu(\emptyset) = 0$ und
	\begin{align*}
		\forall (S_i)_{i \in \nat} \subseteq \mathcal{S}, \text{ paarweise disjunkt und } \biguplus_{i \in \nat}{S-i} \in \mathcal{S} : \quad \mu\left(\biguplus_{i \in \nat}{S_i}\right) = \sum_{i \in \nat}{\mu(S_i)}
	\end{align*}
	Dann existiert eine Fortsetzung von $\mu$ zu einem Maß auf $\sigma(\mathcal{S})$. \\
	Wenn $(G_i)_{i \in \nat} \subseteq \mathcal{S}, G_i \uparrow X, \mu(G_i) < \infty$ gilt, dann ist die Fortsetzung eindeutig.
\end{thm}

\begin{satz}
	$\lambda^1$ ist Prämaß auf $\mathcal{I}$.
\end{satz}
\begin{proof}
	Beweisidee: \\
	($M_1$) ist wegen $\emptyset = [a,a)$ klar. \\
	
\end{proof}

\end{document}

