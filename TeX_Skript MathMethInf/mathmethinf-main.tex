\documentclass[a4paper, 11pt, chapterprefix]{scrreprt}


\KOMAoptions{%
	parskip=half,%
	fontsize=11pt}

%=== PACKAGES ===============================================================
%
% ================================ 
% | PACKAGES                     |
% ================================

\usepackage[top=2.5cm,bottom=2.5cm,left=2.5cm,right=2.5cm]{geometry}

%Schrift und Codierung
\usepackage[T1]{fontenc}
\usepackage[ngerman]{babel}
\usepackage[utf8]{inputenc}

%Mathematik-Umgebungen, Symbole etc.
\usepackage{amsmath}
\usepackage{amssymb}
\usepackage{amsfonts}
\usepackage{mathtools}
\usepackage{latexsym}
\usepackage{marvosym}                    % lighning
\usepackage{bbm}						 % unitary matrix 1
\usepackage{mathrsfs}
\usepackage{stmaryrd}
\usepackage[amsthm,thmmarks,hyperref]{ntheorem}
\usepackage[ntheorem,framemethod=TikZ]{mdframed}


\usepackage{xcolor}						 % Befehl \textcolor für Farbe
\usepackage{graphicx}					 % Befehl \includegraphics für Abbildungen
\usepackage{pgfplots}
\usepackage[german=guillemets]{csquotes}	%Befehl \enquote für Anführungszeichen
\usepackage[breaklinks=true]{hyperref}	 % bessere Unterstützung der PDF-Ausgabe


\usepackage{bbm} %unitary matrix 1
\usepackage{courier}
\usepackage{abstract} % Zusammenfassung am Anfang

%\usepackage[texindy]{imakeidx}
%\makeindex
%\makeindex[name=symbols,title=Symbolverzeichnis]

\usepackage{enumerate}
\usepackage{enumitem} %customize label
%\usepackage{paralist}
\usepackage{pifont}
\usepackage{tabularx}
\usepackage{multirow}
\usepackage{listings}
\usepackage{booktabs}
\usepackage{cleveref}
\usepackage{xfrac}	%sfrac -> fractions e.g. 3/4
%\usepackage{parskip}	%split paragraphs by vspace instead of intendations
%letztes Paket wird durch KOMAoptions ersetzt

\usepackage{cancel}

\usepackage{chngcntr}

%============================================================================
%=== SETTINGS AND NEW DEFINITIONS ===========================================
%============================================================================

% |--------------------------|
% | Farben                   |
% |--------------------------|
\definecolor{lightgrey}{gray}{0.89}
\definecolor{darkgrey}{gray}{0.6}

\definecolor{blue}{rgb}{50,85,150}
\definecolor{lightred}{rgb}{1,0.6,0.6}
\definecolor{darkgreen}{rgb}{0,0.6,0}

% |--------------------------|
% | Zähler                   |
% |--------------------------|
\newcounter{themcount}
\newcounter{defcount}
%\renewcommand{\thethemcount}{\relax}

\numberwithin{themcount}{section}
\counterwithout{section}{chapter}

% |--------------------------|
% | Theoreme                 |
% |--------------------------|
%
% ================================ 
% | THEOREME & UMGEBUNGEN        |
% ================================

% >> Counter für Umgebungen
\newcounter{themcount}
\newcounter{defcount}
\newcounter{taskcount}
\newcounter{blattcount}
%\renewcommand{\thethemcount}{\relax}

% >> Eigenschaften für Übungsblätter
\newcommand{\uebungsgruppe}{Tag x. DS, (un)gerade Woche}
\newcommand{\name}{Eric Kunze}
\newcommand{\matrikelnr}{4679202}
\newcommand{\modul}{\hfill}
\newcommand{\uebungsleiter}{\hfill}

% >> Überschriften
\newcommand{\aufgabe}[1]{{\Large \textsf{\textbf{Aufgabe #1}}}}
\newcommand{\teilaufgabe}[1]{\textsf{\textit{{\large Teilaufgabe (#1)}}}}

% >> Farben
\definecolor{lightgrey}{gray}{0.89}
\definecolor{darkgrey}{gray}{0.6}

\definecolor{myblue}{rgb}{50,85,150}
\definecolor{lightred}{rgb}{1,0.6,0.6}
\definecolor{darkgreen}{rgb}{0,0.6,0}

% >> Theoreme

\newcommand{\boxskip}{7pt}        		% Abstand in Boxen zwischen Rand und Text
\newcommand{\skiparound}{10pt}   		% Abstand vor und nach Theoremen
\newcommand{\thmstyle}{break}           % Sytle für Theorem-Umgebungen

\theoremstyle{\thmstyle}
\theorembodyfont{}                % setzt bodyfont auf nicht kursiv

\theorempreskip{\skiparound}      % Abstände
\theorempostskip{\skiparound}


%---------Theorem-------------
\newmdtheoremenv[%
	backgroundcolor=lightgrey,%
	%linecolor=darkgrey,%
	%innertopmargin=\boxskip,%
	%innerbottommargin=\boxskip,%
	%topline=true,%
	%rightline=true,%
	%leftline=true,%
	%bottomline=true,%
	%innertopmargin=3pt,%
	%innerbottommargin=3pt,%
	leftmargin=-10pt,%
	rightmargin=-10pt,%
	%frametitlefont=\normalfont\bfseries\color{black},%
	%skipabove=5pt,%
	%skipbelow=5pt,%
	skipabove=\skiparound,%
	skipbelow=\skiparound,%
]{thm}[themcount]{Theorem}

%---------Satz----------------
\newmdtheoremenv[%
	%backgroundcolor=lightgrey,%
	%linecolor=darkgrey,%
	%innertopmargin=\boxskip,%
	%innerbottommargin=\boxskip,%
	%linecolor=darkgrey,%
	%topline=true,%
	%rightline=true,%
	%leftline=true,%
	%bottomline=true,%
	%backgroundcolor=lightgrey,%
	%innertopmargin=3pt,%
	%innerbottommargin=3pt,%
	leftmargin=-10pt,%
	rightmargin=-10pt,%
	%frametitlefont=\normalfont\bfseries\color{black},%
	%skipabove=5pt,%
	%skipbelow=5pt,%
	skipabove=\skiparound,%
	skipbelow=\skiparound,%
]{satz}[themcount]{Satz}
%\newtheorem{satz}[themcount]{Satz}

%---------Definition-----------
%\theoremstyle{nonumberbreak}
%\newmdtheoremenv[%
%	outerlinewidth=3pt,%
%	linecolor=darkgrey,%
%	topline=false,%
%	rightline=false,%
%	bottomline=false,%
%	innertopmargin=3pt,%
%	innerbottommargin=3pt,%
%	frametitlefont=\normalfont\bfseries\color{black},%
%	skipabove=5pt,%
%	skipbelow=5pt,%
%]{defin}[themcount]{Definition}
%\theoremstyle{\thmstyle}
\newtheorem{defin}[themcount]{Definition}


%---------Lemma---------------
\newtheorem{lemma}[themcount]{Lemma}

%---------Folgerung--------------
\newtheorem{folg}[themcount]{Folgerung}

%---------Korollar--------------
\newtheorem{kor}[themcount]{Korollar}

%---------Beispiel----------------
\newtheorem{bsp}[themcount]{Beispiel}

%---------Erinnerung--------------
\newtheorem{erinnerung}[themcount]{Erinnerung}

%---------Bemerkung--------------
%\theoremstyle{nonumberplain}
\newtheorem{bem}[themcount]{Bemerkung}

\theoremstyle{nonumberplain}
\theorembodyfont{\normalfont}
\newtheorem{bsppure}{Beispiel}

%--------------------------------
\theoremstyle{break}

\theoremheaderfont{\sffamily\bfseries\large}
%
%--------------------------------
%
%----------Übung-----------------
\newmdtheoremenv[%
%backgroundcolor=lightgrey,%
%linecolor=darkgrey,%
%innertopmargin=\boxskip,%
%innerbottommargin=\boxskip,%
%topline=true,%
%rightline=true,%
%leftline=true,%
%bottomline=true,%
%innertopmargin=3pt,%
%innerbottommargin=3pt,%
%leftmargin=-10pt,%
%rightmargin=-10pt,%
%frametitlefont=\sffamily\bfseries\color{black},%
%skipabove=5pt,%
%skipbelow=5pt,%
skipabove=\skiparound,%
skipbelow=\skiparound,%
]{uebung}[taskcount]{Übung}
%
%-----------Lösung-----------------
\newenvironment{loesung}{%
	\vspace{-2mm}
	\textit{Lösung:} \\%
}{}



%---------Übungsblatt--------------
\newenvironment{uebungsblatt}{%
	\stepcounter{blattcount}%
	\setcounter{page}{1}
	\fcolorbox{black}{lightgrey}{%
		\begin{minipage}{0.5\textwidth}
			\textsf{{\huge \textbf{Übungsblatt \theblattcount}}} \\
			\textsf{\modul}
		\end{minipage}
		\begin{minipage}{0.5\textwidth}
			\flushright \name \\
			Übungsleiter: \uebungsleiter
	\end{minipage}}%
	\setcounter{figure}{0}%
	\setcounter{equation}{0}%
	\setcounter{table}{0}%
	\vspace{5mm}%
}{%
	\pagebreak%
}

%---------Hausaufgaben--------------
\newenvironment{hausaufgabe}{%
	\stepcounter{blattcount}%
	\setcounter{page}{1}
	\fcolorbox{black}{lightgrey}{%
		\begin{minipage}{0.5\textwidth}
			\textsf{{\huge \textbf{Hausaufgaben}}} \\
			\textsf{Übungsblatt \theblattcount}
		\end{minipage}
		\begin{minipage}{0.5\textwidth}
			\flushright \textbf{\name} (Matr.-Nr. \matrikelnr)\\
			Ü-Gruppe: \uebungsgruppe
		\end{minipage}}%
	\setcounter{figure}{0}%
	\setcounter{equation}{0}%
	\setcounter{table}{0}%
	\vspace{5mm}%
	}{%
	\pagebreak%
}

%
%
%\newcommand{\headeraufgabenblatt}{%
%	\stepcounter{blattcount}%
%	\fcolorbox{black}{lightgrey}{%
%		\begin{minipage}{0.5\textwidth}
%			\textsf{{\huge \textbf{Hausaufgaben}}} \\
%			\textsf{Übungsblatt \theblattcount}
%		\end{minipage}
%		\begin{minipage}{0.5\textwidth}
%			\flushright \textbf{Eric Kunze} (Matr.-Nr. 4679202)\\
%			Ü-Gruppe: gerade Woche (M. Schönherr)
%		\end{minipage}%
%	} \setcounter{figure}{0} \setcounter{equation}{0} \\[5mm]}



% |--------------------------|
% | Kapitelüberschriften     |
% |--------------------------|
\renewcommand{\thechapter}{\Roman{chapter}}
%\usepackage{titlesec}
%\titleformat{\chapter}{\Large}{\texttt{\thechapter .~ \chaptertitlename}}{1em}{}{}
%\renewcommand{\chapterformat}{\Large}
%
%\usepackage{mwe}
%\renewcommand{\chapterlineswithprefixformat}[3]{%
%  #2 \nobreak
%  #3 \nobreak
%  \rule{\textwidth}{1pt}%
%}

% |--------------------------|
% | neue Befehle             |
% |--------------------------|
%============= BUCHSTABEN =============
\renewcommand{\epsilon}{\varepsilon}
\renewcommand{\phi}{\varphi}

%============= LOGIK =============
\newcommand{\follows}{\quad \Rightarrow \quad}
\newcommand{\equivalent}{\quad \Leftrightarrow \quad}

%============= MENGEN =============
\newcommand{\nat}{\mathbb{N}}				% natürliche Zahlen
\newcommand{\integer}{\mathbb{Z}}			% ganze Zahlen
\newcommand{\real}{\mathbb{R}}				% reelle Zahlen
\newcommand{\complex}{\mathbb{C}}			% komplexe Zahlen
\newcommand{\korper}{\mathbb{K}}			% Körper IR oder IC
\newcommand{\ringx}{R^{\times}}				% Einheiten Ring
\newcommand{\korperx}{K^{\times}}			% Einheiten Körper
\newcommand{\rest}[1]{\integer / #1 \integer} % Restklassen Z/( )Z
\newcommand{\ohneNull}{\backslash \menge{0}}% Menge ohne Null \{0}
\newcommand{\einheit}[1]{{#1}^{\times}}

\newcommand{\menge}[1]{\{ #1 \}}
\newcommand{\card}[1]{\abs{#1}}
\newcommand{\st}{\,|\,}
\DeclareMathOperator{\inn}{int} % Set of inner points
\DeclareMathOperator{\ext}{ext} % Set of outer points
\DeclareMathOperator{\cl}{cl} 	% Abschluss
\DeclareMathOperator{\diam}{diam}

%============= FUNKTIONEN =============
\newcommand{\abb}[3]{#1 \colon #2 \to #3}
\DeclareMathOperator{\id}{id}
\DeclareMathOperator{\graph}{graph}
\DeclareMathOperator{\grad}{grad}
\newcommand{\quer}[1]{\bar{#1}}

%============= ZAHLENTHEORIE =============
\newcommand{\teilt}{\mid}
\newcommand{\normalteiler}{\trianglelefteq}
\newcommand{\nichtnormal}{\ntrianglelefteq}
\DeclareMathOperator{\ggT}{ggT}
\DeclareMathOperator{\kgV}{kgV}

\newcommand{\simfrac}[1]{\frac{1}{#1}}

%============= VEKTORRÄUME =============
\newcommand{\basisB}{\mathcal{B}}
\newcommand{\darstMat}[2]{M_{\mathcal{#1}} (#2)}
\newcommand{\MBf}{\darstMat{\basisB}{f}}
\newcommand{\one}{\mathbbm{1}}

\newcommand{\isomorph}{\cong}

\newcommand{\erz}[1]{\langle #1 \rangle}
\newcommand{\scal}[2]{\langle #1\, ,\, #2 \rangle}

\DeclareMathOperator{\Span}{span}

\DeclareMathOperator{\Hom}{Hom}
\DeclareMathOperator{\End}{End}
\DeclareMathOperator{\Aut}{Aut}
\DeclareMathOperator{\Ker}{Ker}
\DeclareMathOperator{\bild}{Im}

\DeclareMathOperator{\Mat}{Mat}
\DeclareMathOperator{\diag}{diag}
\DeclareMathOperator{\GL}{GL}
\DeclareMathOperator{\SL}{SL}
\DeclareMathOperator{\rang}{rang}
\DeclareMathOperator{\tr}{tr}
\newcommand{\transpose}[1]{\left( #1 \right)^{\top}}

\DeclareMathOperator{\sgn}{sgn}
\DeclareMathOperator{\Sym}{Sym}

\DeclareMathOperator{\Eig}{Eig}

%============= GRUPPEN =============
\DeclareMathOperator{\ord}{ord}
\DeclareMathOperator{\Inn}{Inn}
\DeclareMathOperator{\zentrum}{Z}
\DeclareMathOperator{\fix}{Fix}
\DeclareMathOperator{\stab}{Stab}

%============= RINGE =============

%============= METRISCHE RÄUME =============
\newcommand{\abs}[1]{\left| #1 \right|}
\newcommand{\norm}[1]{\left\lVert #1 \right\rVert}

%============= DIFFERENZIERUNG =============
\newcommand{\partdiff}[1]{\frac{\partial}{\partial #1 }}
\newcommand{\diff}[1]{\enspace \mathrm{d}#1}
\newcommand{\dx}{\diff{x}}
\newcommand{\dy}{\diff{y}}
\newcommand{\da}{\diff{a}}

\DeclareMathOperator{\Lin}{L}		% Menge der lineare Abb.
\DeclareMathOperator{\cdiff}{C}		% Menge der stetig-diffbaren Abb.

%============= INTEGRATION =============
\DeclareMathOperator{\tangentialraum}{T}
\DeclareMathOperator{\normalenraum}{N}
\DeclareMathOperator{\divergenz}{div}
\renewcommand{\div}{\divergenz}
\DeclareMathOperator{\rot}{rot}
\newcommand{\rand}{\partial}

\newcommand{\bigabb}[5]{#1 \colon \left\lbrace%
\begin{array}{ccl}%
	#2 & \to & #3 \\%
	#4 & \mapsto & #5%
\end{array}%
\right.}
%
\newcommand{\bigabbnoname}[4]{\left\lbrace%
\begin{array}{ccl}%
	#1 & \to & #2 \\%
	#3 & \mapsto & #4%
\end{array}%
\right.}



\DeclareMathSymbol{*}{\mathbin}{symbols}{"01}

\renewcommand{\thefootnote}{\alph{footnote}}

\renewcommand*{\theequation}{\arabic{equation}}
\renewcommand{\labelitemi}{--}
\renewcommand{\labelenumi}{\alph{enumi})}


\begin{document}

\setlength\abovedisplayshortskip{3pt}
\setlength\belowdisplayshortskip{3pt}
\setlength\abovedisplayskip{3pt}
\setlength\belowdisplayskip{3pt}

\section{Zyklische Gruppen}
\begin{defin}[Zyklische Gruppen der Ordnung n]
	$Z_n := \erz{a \mid a^n = e} = \{a^0,a^1, \dots a^{n-1}\}$
\end{defin}
\begin{lemma}
	$Z_n$ ist isomorph zu $\rest{n}$, d.h. es existiert ein Isomorphismus $f: \rest{n} \to Z_n, i \mapsto a^i$.
\end{lemma}
\begin{proof}
	\begin{enumerate}
		\item $f$ ist bijektiv: Es genügt zu zeigen, dass $f$ injektiv ist. \\
		\begin{align*}
		f(a^i) = f(a^j) \follows i + n\integer = j +n\integer \overset{i,j \in \rest{n}}{\follows} i=j \follows f \text{ bijektiv}
		\end{align*}
		\item $f(i+j) = f(i) + f(j) \forall i,j \in \rest{n}$ \\
		\begin{align*}
		f(i+j) = a^{i+j} = a^i * a^j = f(i) * f(j)
		\end{align*}
	\end{enumerate}
\end{proof}

\begin{bem}[Eigenschaften von $Z_n$]
	\begin{itemize}
		\item $Z_n$ ist abelsch.
		\item Zu jedem Teiler $t$ von $n$ gibt es genau eine Untergruppe der Ordung $t$, nämlich $\erz{a^{\frac{n}{t}}}$.
		\item Untergruppen von zyklischen Gruppen sind wieder zyklisch.
	\end{itemize}
\end{bem}

\begin{lemma}
	Sei $(G,\circ)$ eine zyklische Gruppe der Ordnung $n$ mit $G=\erz{n}$. Sei weiter $U$ eine Untergruppe von $G$. Dann ist $U$ zyklisch, d.h. es gibt ein Element $a^k$ mit $U=\erz{a^k}$.
\end{lemma}
\begin{proof} Wir zerlegen die Behauptung in zwei Fälle.
	\begin{enumerate}
		\item Ist $\# U = 1$, d.h. $U=\{e=a^0\}$ ist zyklisch.
		\item Sei $\# U > 1$. Somit enthält $U$ ein Element $a^i$ mit $i>0, i$ minimal. Wir zeigen, dass $U=\erz{a^i}$. \\
		Sei $a^j \in U$ beliebig. Dann gilt $a^j \in \erz{a^i}$, denn: \\
		Es gibt $q,r \in \nat$ mit $j = q*i + r$ und $0 \leq r < i$. Dann ist $a^j = a^{q*i + r} = (a^i)^q * a^r$ mit $a^i, a^j \in U$ und somit auch $(a^i)^q \in U$ sowie schlussendlich auch $a^r \in U$. Da $i$ minimal ist, folgt $r=0$ und dann $a^r=e$, sodass $a^j = (a^i)^q * e = (a^i)^q \in \erz{a^i}$
	\end{enumerate}
\end{proof}

\begin{defin}
	Seien $(G_1,\circ_1), (G_2, \circ_2)$ Gruppen und $g_1, g_1' \in G_1$ und $g_2 , g_2' \in G_2$. Durch 
	\begin{align*}
	(g_1,g_2) \circ (g_1',g_2') = (g_1 \circ_1 g_1' , g_2 \circ_2 g_2')
	\end{align*}
	wird eine Operation in $G_1 \times G_2$ erklärt. Man nennt $(G_1 \times G_2 , \circ)$ das direkte Produkt der Gruppen $G_1$ und $G_2$.
\end{defin}
\begin{bem}
	Offensichtlich ist $(G_1 \times G_2 , \circ)$ eine Gruppe.
\end{bem}
%
\begin{satz}
	$(G_1, \circ_1), (G_2 , \circ_2)$ seien Gruppen.
	\begin{enumerate}
		\item $G_1 \times G_2 \isomorph G_2 \times G_1$
		\item Sind $G_1$ und $G_2$ abelsch, so ist auch $G_1 \times G_2$ abelsch.
		\item Ist $G_1 \times G_2$ zyklisch, so sind auch $G_1$ und $G_2$ zyklisch.
	\end{enumerate}
\end{satz}

\begin{bsp}
	$\rest{2} \times \rest{2} \neq \rest{4}$ \\
	$\rest{2} \times \rest{3} \isomorph \rest{6}$, denn $\erz{(1,1)} = \{ (1,1) , (0,2) , (1,0) , (0,1) ,(1,2) , (0,0) \}$.
\end{bsp}

\begin{satz}
	Die Gruppe $\rest{n} \times \rest{m}$ ist genau dann zyklisch, wenn $\ggT(n,m) = 1$.
\end{satz}
\begin{proof}
	$\ggT(n,m)=1 \follows \rest{n} \times \rest{m} = \rest{n*m} = \erz{(1,1)}$ \\
	Sei $\ggT(n,m) = d > 1$ und $(a,b) \in \rest{n} \times \rest{m}$. Dann ist $\ord(a,b) = \# \erz{(a,b)} < n*m = \# \rest(n) \times \rest{m}$. \\
	Sei nun $n=n' * d$ und $m=m' * d$. Dann ist
	\begin{align*}
	\underbrace{(a,b) + \cdots (a,b)}_{n'*m'*d < n*m \text{ Summanden}} = (0,0)
	\end{align*}
\end{proof}

\begin{thm}[Basissatz für endliche abelsche Gruppen]
	Jede endliche abelsche Gruppe ist isomorph zu einem direkten Produkt zyklischer Gruppen von Primzahlpotenzordnung
	\begin{align*}
	Z_{m_1} \times Z_{m_2} \times \cdots \times Z_{m_k} \quad \text{mit } m_1 \teilt m_2, m_2 \teilt m_3 , \dots , m_{k-1} \teilt m_k
	\end{align*}
	Diese Darstellung ist eindeutig bis auf die Reihenfolge der Faktoren im direkten Produkt.
\end{thm}
\begin{bsp}
	Suche alle abelschen Gruppen der Ordnung 8.\\
	$8=2^3=2^1*2^1*2^1$ \\
	$Z_8 = Z_{2^3}$ \\
	$Z_{2^2} \times Z_{2^1} = Z_4 \times Z_2$ \\
	$Z_{2^1} \times Z_{2^1} \times Z_{2^1} = Z_2 \times Z_2 \times Z_2$\\
	$\follows$ Es gibt bis auf Isomorphie genau 3 abelsche Gruppen der Ordnung 8.
\end{bsp}
\begin{bsp}
	Alle abelschen Gruppen der Ordnung 360 enthalten ein Element der Ordnung 30. \\
	$360 = 2^3 * 3^2 * 5$
\end{bsp}

\newpage
\section{Ringe}
\section{Ringe}

\begin{defin}
	Sei $R \neq \emptyset$. $(R, + , *)$ heißt \begriff{Ring}, falls gilt:
	\begin{enumerate}
		\item $(R,+)$ ist eine abelsche Gruppe.
		\item $R,*)$ ist eine Halbgruppe.
		\item Distributivgesetze: $a*(b+c) = (a*b)+(a*c)$ und $(b+c)*a = (b*a) + (c*a)$ für alle $a,b,c \in R$.
		\item Gilt zusätzlich \\
		$a*b = b*a$ für alle $a,b \in R$, \\
		dann wird $(R+,*)$ kommutativer Ring genannt.
	\end{enumerate}
\end{defin}
%
\begin{defin}
	Sei $(R,+,*)$ ein Ring und $U \subseteq R$. $U$ heißt \begriff{Unterring} von $(R,+,*)$, wenn gilt:
	\begin{enumerate}
		\item $U \neq \emptyset$ ($0_R \in U$)
		\item $a,b, \in U \follows a+b \in U$ für alle $a,b \in U$ (Abgschlossenheit unter Addition)
		\item $a \in U \follows -a \in U$ für alle $a \in U$ (Abgeschlossenheit unter additiven Inversen)
	\end{enumerate}
\end{defin}
%
\begin{bsp}
	$\integer \subseteq \mathbb{Q} \subseteq \real \subseteq \complex$ sind kommutative Ringe. \\
	$\real^{n \times n}$, der Matrizenring (über $\real$) \\
	$\rest{n}$, der Restklassenring modulo $n$ \\
	$2\integer = \menge{2*z : z \in \integer}$ ist ein Unterring von $\integer$
	$\menge{a+bi : a,b \in \integer}$ ist Unterring von $\complex$
\end{bsp}
%
\begin{bem}
	Allgemein gilt:
	\begin{align*}
		a * (b_1 + \cdots + b_n) = a*b_1 + \cdots + a*b_n
	\end{align*}
	für alle $a,b_i \in R$.
\end{bem}
\begin{proof}
	Zeige die Aussage mittels vollständiger Induktion über $n$.
\end{proof}
%
\begin{bem}
	Addition ist in jedem Ring kommutativ. \\
	"Punktrechnung vor Strichrechnung." \\
	Inverse Elemente in Ringen existieren immer bzgl. der Addition (Bezeichnung $-a$), und sofern sie bzgl. der Multiplikation existieren schreibe $a^{-1}$.
\end{bem}
%
\begin{bem}
	Jeder Ring hat ein neutrales Element bezüglich der Addition. Nenne dieses auch \begriff{Nullelement} und bezeichne es mit $0$. \\
	Das Nullelement ist eindeutig bestimmt, denn: $0_1 = 0_1 + 0_2 = 0_2$.
\end{bem}
%
\begin{defin}
	Sei $(R,+,*)$ ein Ring mit Nullelement $0$. Existiert ein Element $1 \in R \backslash \menge{0}$ mit $a*1 = 1*a = 1$ für alle $a \in R$, dann wird $1$ \begriff{Einselement} genannt.
\end{defin}
%
\begin{bem}
	Nicht jeder Ring hat ein Einselement! Falls ein solches aber existiert, dann ist es auch eindeutig bestimmt, denn: $1_1 = 1_1 * 1_2 = 1_2$.
\end{bem}
%
\begin{bsp}
	\begin{itemize}
		\item $\integer, \mathbb{Q}, \real, \complex$ sind Ringe mit Nullelement $0$ und Einselement $1$.
		\item $\rest{n}$ ist ein Ring mit Nullelement $0$ und Einselement $1$.
		\item $\real^{n \times n}$ ist ein Ring mit Nullelement $0_{n \times n}$ und Einselement $\one_n$.
		\item Sei $M \neq \emptyset$ und $(\mathcal{P}(M); \triangle, \cap)$ ist dann ein Ring mit Nullelement $\emptyset$ und Einselement $M$.
	\end{itemize}
\end{bsp}
%
\begin{bem}
	Sei $(R,+,*)$ ein Ring mit Nullelement $0$ und $a \in R$. Dann gilt $0 * a = 0$ und $a * 0 = 0$.
\end{bem}
\begin{proof}
	$0*a = (0 + 0)*a = (0 * a) + (0 * a) \follows (0*a) + (-0*a) = (0*a) + (0*a) + (-0*a) \follows 0 = 0*a + 0 = 0*a$
\end{proof}
%
\begin{defin}
	Sei $(R,+,*)$ ein kommutativer Ring mit $a,b \in R \backslash \menge{0}$. Gilt $a*b = 0$, dann werden $a,b$ \begriff{Nullteiler} in $(R,+,*)$ genannt.
\end{defin}
%
\begin{bsp}
	Der Ring $\rest{6}$ hat die Nullteiler $2$ und $3$, denn $2*3=3*2=0$. \\
	Die Ringe $\integer, \mathbb{Q}, \real, \complex$ besitzen keine Nullteiler, sind also nullteilerfrei. \\
	In Matrizenringen gibt es Nullteiler, z.B.
	\begin{align*}
		\begin{pmatrix} 1 & 0 \\ 0 & 0 \end{pmatrix} * 
		\begin{pmatrix} 0 & 0 \\ 0 & 1 \end{pmatrix} =
		\begin{pmatrix} 0 & 0 \\ 0 & 0 \end{pmatrix}
	\end{align*}
	In $\rest{p}$ mit $p$ prim gibt es keine Nullteiler, denn: Sei $a \in \rest{p} \backslash \menge{0}$. Angenommen es existiert ein $b \in \rest{p} \backslash \menge{0}$ mit $a*b=0 \enspace (\mod p)$. Dann folgt $(a^{-1}*a)*b = a^{-1} * 0$ und $1*b=b=0$. \Lightning
\end{bsp}
%
\begin{defin}
	Sei $(R,+,*)$ ein kommutativer Ring mit Einselement, in dem es keine Nullteiler gibt (nullteiilerfrei). Dann wird $(R,+,*)$ ein \begriff{Integritätsring} genannt.
\end{defin}
%
\begin{bsp}
	$(\integer, +,*)$ ist ein Integritätsring.
\end{bsp}
%
\begin{defin}
	Sei $(R,+,*)$ ein Ring mit Nullelement $0$. Ist $(R \backslash \menge{0}), +,*)$ eine abelsche Gruppe, dann nennt man $(R,+,*)$ einen \begriff{Körper}.
\end{defin}

\end{document}

