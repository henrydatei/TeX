\section{Fourier-Transformationen}
\subsection{Diskrete Fourier-Transformation (DFT)}
\begin{bsp}
	Sei $p(x) = 1+2x+0x^2+x^3$. Auwertung an $\omega^0, \omega^1, \omega^2, \omega^3$ einer primitiven $4$-ten Einheitswurzel $\omega$ in $\complex$. Wähle $\omega = i$, also an $1, i, -1, -i$
	\begin{align*}
		DFT_{4,i} = 
		\begin{pmatrix}
		1 & 1 & 1 & 1 \\
		1 & i & -1 & -i \\
		1 & -1 & 1 & -1 \\
		1 & -i & -1 & i
		\end{pmatrix}
	\end{align*}
	\begin{align*}
		DFT_{4,i} * \begin{pmatrix} 1 \\ 2 \\ 0 \\ 1 \end{pmatrix}
		= \begin{pmatrix}
		4 \\ 1+i \\ -2 \\ 1-i
		\end{pmatrix}
	\end{align*}
	Also ist $p(1) = 4$, $p(i) = 1+i$, $p(-1) = -2$, $p(-i) = 1-i$.
\end{bsp}

\subsection{Schnelle Fourier-Transformation (FFT)}
\begin{bem}[Polynommultiplikation]
	Gegeben seine Polynome $a, b$ und wir suchen das Polynom $a*b$. Wende zuerst DFT auf beide Polynome $a$ und $b$ an. Mit komponentenweiser Multiplikation im Körper und inverser DFT erhält man $a(x)*b(x)$.
\end{bem}

\begin{bem}[Multiplikation großer natürlicher Zahlen]
	Gegeben seien zwei natürliche Zahlen $a, b \in \nat$.
	\begin{align*}
		a &= a_0 + a_1 * 2 + a_2 * 2^2 + \cdots + a_n * 2^n = \sum_{i=0}^n{a_i*2^i} \\
		b &= b_0 + b_1 * 2 + b_2 * 2^2 + \cdots + b_n * 2^n = \sum_{i=0}^n{b_i*2^i}
	\end{align*}
	Daraus basteln wir nun Polynomfunktionen 
	\begin{align*}
		a(x) &= a_0 + a_1 * x + a_2 * x^2 + \cdots + a_n*x^n \\
		b(x) &= b_0 + b_1 * x + b_2 * x^2 + \cdots + b_n*x^n \\
	\end{align*}
	Mit DFT erhält man dann $a(x) * b(x) = p(x)$, und für die Multiplikation
	\begin{align*}
		a*b = p(x) = p(2) = a(2) * b(2)
	\end{align*}
\end{bem}

\begin{bem}
	$\omega$ ist primitive $(n+1)$-te Einheitswurzel $\follows$ $\omega^2$ ist primitive $\frac{n+1}{2}$-te Einheitswurzel.
\end{bem}
\begin{proof}
	$\omega$ ist eine primitive $(n+1)$-te Einheitswurzel, wenn
	\begin{itemize}
		\item $\omega^{n+1} = 1$ und
		\item $\card{\erz{\omega}} = \card{\menge{\omega^0, \omega^1, \omega^2, \dots, \omega^n}} = n+1$.
	\end{itemize}
	Damit ist die Aussage klar.
\end{proof}
\begin{bem}
	$\omega^k$ ist primitive $(n+1)$-te Einheitswurzel \\
	$\follows$ $-\omega^k$ ist primitive $(n+1)$-te Einheitswurzel, falls $\omega$ primitive $(n+1)$-te Einheitswurzel ist.
\end{bem}
\begin{proof}
	$(\omega^k)^{n+1} = 1 \follows (-\omega^k)^{n+1} = (-1)^{n+1} * (\omega^k)^{n+1} = 1 * 1$, da $n+1 = 2^r$ vorausgesetzt war.
\end{proof}

\begin{satz}[Aufwand $A(r)$ für Algorithmus von Cooley-Tukey]
	Seien $n+1 = 2^r$, $\omega$ eine primitive $(n+1)$-te Einheitswurzel und
	\begin{align*}
		p(x) = a_0 * a_1 * x + \cdots + a_n * x^n = (a_0 + a_2 * y + \cdots) + x*(a_1 + a_3 * y + \cdots) \text{ mit } y = x^2
	\end{align*}
	Dabei wird der linke Summand an den Stellen $\omega^0, \dots, \omega^n$ und der rechte Summand an \\
	$(\omega^2)^0, \dots, (\omega^2)^{\frac{n+1}{2}}$ ausgewertet wird.
\end{satz}
\begin{bsp}
	\begin{itemize}
		\item $r=0 \follows n+1 = 1 \follows n = 0 \follows p(x) = a_0 \follows A(r) = 0$
		\item $r=1 \follows n+1 = 2 \follows n = 1 \follows p(x) = a_0 + a_1 * x \follows A(1) = 2*0 + 2 +1 = 3$. \\
		\item $r=2 \follows n+1 = 4 \follows n = 3$ \\
		$\follows p(x) = a_0 + a_1 * x + a_2 * x^2 + a_3 * x^3 = (a_0 + a_2 * y) + x * (a_1 + a_3 * y)$ mit $y=x^2$.
		$\follows A(2) = 2*3 + 4 + 2 = 12$
	\end{itemize}
	Allgemein ist damit
	\begin{align*}
		A(r) = 2 * A(r-1) + \underbrace{2^r}_{\text{Addition}} + \underbrace{2^{r-1}}_{\text{Multiplikation}}
	\end{align*}
\end{bsp}
\begin{satz}
	Für den Aufwand gilt
	\begin{align*}
		A(r) = \frac{3}{2} * r * 2^r
	\end{align*}
\end{satz}
\begin{proof}
	Für $r=0$ ist $A(0) = 0$ \checkmark \\
	Für $r>0$ ist 
	\begin{align*}
		A(r) 
		&= 2 * A(r-1) + 2^r + 2^{r+1} = 2 * \frac{3}{2} (r-1) *2^(r-1) + 2^r + 2^{r+1} \\
		&= \frac{3}{2} * r * 2^r - \frac{3}{2} * 2^r + 2^r + 2^{r+1} \\
		&= \frac{3}{2} * r * 2^r
	\end{align*}
\end{proof}

\begin{bem}
	Außerdem erhält man
	\begin{align*}
		n+1 = 2^r \follows r = \log_2(n+1) \follows \frac{3}{2} * \log_2(n+1) (n+1)
	\end{align*} 
	und damit ist der Aufwand für FFT $\mathcal{O}(n * \log(n))$.
\end{bem}
