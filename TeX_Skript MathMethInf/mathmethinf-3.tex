\section{Endliche Körper}

\begin{defin}
	Ein endlicher Körper mit $p^k$ Elementen ($p$ prim, $k \in \nat \backslash \menge{0,1}$), in Zeichen $\GF(p^k)$ ist $(\GF(p^k)[x] / f(x) , \oplus, \otimes)$ mit $f(x) \in \GF(p)[x], \grad(f) = k$ und $f$ ist irreduzibel.
\end{defin}

\begin{defin}
	Sei K ein Körper, $f(x) \in K[x]$. Dann heißt $f(x)$ irreduzibel über $K$, wenn es keine Polynome $a(x), b(x) \in K[x]$ gibt, sodass $f(x)=a(x) * b(x)$ und $0 < \grad(a(x)) \leq \grad(b(x)) < \grad(f(x))$ gilt.
\end{defin}

\begin{bsp}
	\begin{itemize}
		\item Sei $K = \GF(2)$, dann ist $x^4+1 = (x-1)^4$ nicht irreduzibel.
		\item Sei $K = \GF(2)$, dann ist $x^3 + x + 1$ irreduzibel.
		\item Sei $K = \GF(2)$, dann sind $x^4+x^3+1$, $x^4+x+1$, $x^4+x^3+x^2+x+1$ irreduzibel.
	\end{itemize}
\end{bsp}

\begin{bem}
	Aber für $K = \GF(3)$ ist $2x+1 = 2x + 1$ ist irreduzibel, d.h. konstante Faktore können immer ausgeklammert werden ohne die Irreduzibiliät zu verändern.
\end{bem}

\begin{bsp}
	Wir betrachten einen endlichen Körper mit $2^4$ Elementen. Dann ist $\GF(2) = \rest{2} = \menge{0,1}$. 
	Menge
	\begin{align*}
	\GF(2)[x] / (x^4+x^3+1) 
	&= \menge{a_3x^3 + a_2x^2 + a_1x + a_0 : a_0, a_1, a_2, a_3 \in \GF(2)} \\
	&= \menge{ \underbrace{0,1}_{\text{Grad 0}}, \underbrace{x,x+1}_{\text{Grad 1}}, \underbrace{x^2, x^2 + 1, x^2 + x, x^2+x+1}_{\text{Grad 2}}, \underbrace{x^3, \dots, x^3+x^2+x+1}_{\text{8 Polynome vom Grad 3}}}
	\end{align*}
\end{bsp}

\begin{bem}
	$\GF(2)[x] / (x^4+x^3+1) = \GF(2)[x] / (x^4+x+1) = \GF(2) [x] / (f(x))$ mit $\grad f(x) = 4$ über $\GF(2)$.
\end{bem}

\begin{bem}
	$a(x)^{-1} = (x^3+x+1)1{-1} = x^3+x$ in $\GF(2)[x]$. \\
	Berechnung von $a(x)^{-1}$ mit erweitertem euklidischen Algorithmus:
	\begin{align*}
		1 \isomorph \ggT(a(x), \underbrace{(x^4+x^3+1)}_{\text{irreduzibel}}) = \alpha(x) \odot a(x) + \beta(x) \odot (x^4+x^3+1)
	\end{align*}
	$\mod (x^4 + x^3 + 1)$:
	\begin{align*}
		&1 \equiv \alpha(x) \odot  a(x) \mod (x^4+x^3+1) \\
		\follows &a(x)^{-1} = \alpha(x) \mod (x^3+x^3+1)
	\end{align*}
\end{bem}

\begin{defin}
	Seo $K = \GF(p)$ ein Körper, $f(x) \in K[x]$ irreduzibel. $f(x)$ heißt \begriff{primitiv}, wenn gilt
	\begin{align*}
		\min \menge{l \in \nat\ohneNull : x^l \equiv 1 (\mod f(x))} = p^k -1
	\end{align*}
\end{defin}

\newpage
\begin{bsp}
	Sei $K = \GF(2)$ und $f(x) = x^3 + x + 1$.
	\begin{table}[h]
		\centering
		\begin{tabular}{|c|l|}
			\hline 
			$l$ & $x^l \mod (x^3 + x + 1)$ \\ 
			\hline 
			$0$ & $1$ \\ 
			\hline 
			$0$ & $x$ \\ 
			\hline 
			$2$ & $x^2$  \\ 
			\hline 
			$3$ & $x^3 = x+1$  \\ 
			\hline 
			$4$ & $x^4 = x(x+1) = x^2 + x$  \\ 
			\hline 
			$5$ & $x^5 = x(x^2+x) = x^3 + x^2 = x^2 + x + 1$ \\
			\hline
			$6$ & $x^6 = x^2 +1$ \\
			\hline
			$7$ & $x^7 = 1$ \\
			\hline
		\end{tabular} 
	\end{table}
	Also ist $\min \menge{\dots} = 7 = 2^3 - 1$ und damit $x^3+x+1$ primitiv über $\GF(2)$
\end{bsp}

\begin{bem}
	Ist $f(x) \in \GF(p)[x]$ mit $\grad(f(x)) = k$ ein primitives Polynom, dann kann man alle Elemente von $\GF(p)[x] / f(x) \ohneNull$ in der Form $x^l \mod f(x)$ notieren, wobei $k \in \menge{0,1,\dots, p^k-2}$. Man stellt fest, dass sowohl $\GF(p)[x] / f(x) \ohneNull$ als auch $x^l \mod f(x)$ genau $p^k-1$ Elemente besitzen ($\GF(p)[x]$ hat $p^k$ Elemente).
\end{bem}