\section{Ordnung und Index}

Sei $G$ eine Gruppe und $g \in G$.

\begin{defin}
	\begin{enumerate}
		\item $\# G = \card{G} \in \nat \cup \{\infty\}$, die Ordnung von $G$.
		\item $\ord(g) = \# \erz{g}$, die Ordnung von $g$.
	\end{enumerate}
\end{defin}
\begin{bsp}
	$\# S_n = n! \quad , \quad \# A_n = \simfrac{2}*n! \quad (n\geq 2)$ \\
	$\# \rest{n} = n$
\end{bsp}

\begin{lemma}
	Für $X \subseteq G$ ist
	\begin{align*}
		\erz{X} = \{ g_1^{\epsilon_1}  * \cdots * g_r^{\epsilon_r} : r \in \nat_0, g_1,\dots,g_r \in X, \epsilon_1 ,\dots, \epsilon_r \in \{1,-1\} \}
	\end{align*}
\end{lemma}
\begin{proof}
	klar, da rechte Seite Untergruppe ist, die $X$ enthält, und jede solche enthält alle  Ausdrücke der Form $g_1^{\epsilon_1}, \cdots g_r^{\epsilon_r}$.
\end{proof}

\begin{satz}
	\begin{enumerate}
		\item Ist $\ord(g)= \infty$, so ist $\erz{g} = \{ \dots , g^{-2}, g^{-1}, 1, g, g^2, \dots \}$.
		\item Ist $\ord(g)=n < \infty$, so ist $\erz{g}=\{1,g,g^2, \dots , g^{n-1} \}$.
		\item Es ist $\ord(g) = \inf\{ k \in \nat : g^k=1 \}$.
	\end{enumerate}
\end{satz}
\begin{proof}
	Nach 2.3 ist $\erz{g}= \{ g^k : k \in \integer \}$. Sei $m=\inf\{k \in \nat : g^k = 1 \}$.
	\begin{itemize}
		\item $| \{ g^k : 0 \leq k < m \} | = m$: Sind $g^a=g^b$ mit $0 \leq a < b < m$, so ist $g^{b-a}=1$, aber $0 < b-a < m$ im Widerspruch zur Minimalität von m.
		\item $m=\infty$: $\follows \ord(g)=\infty \quad$ \checkmark
		\item $m<\infty$: $\follows \erz{g} = \{ g^k : 0 \leq k < m \}:$ Die Inklusion $\{ g^k :0 \leq k < m \} \subseteq \erz{g}$ ist klar. Für die andere Inklusion schreibe $k \in \integer$ als $k=g*m+r$ mit $q,r \in \integer, 0 \leq r < m$. \\ $\follows g^k =g^{q*m+r} = (g^m)^q*g^1 =1^q*g^r =g^r \in \{ 1,g,\dots,g^{m-1} \}$ \\
		$\follows \erz{g} \subseteq  \{ g^k : 0\leq k < m \}$
	\end{itemize}
\end{proof}

\begin{bsp}
	Sei $\sigma \in S_n$ ein $k$-Zykel, so ist $\ord(\sigma)=k$. \\ \textcolor{gray}{(Man muss genau k-mal tauschen bis alle Elemente wieder an ihrem Platz sind)} \\
	Für $\bar{1} \in \rest{n}$ ist $\ord(\bar{1})=n$. \\ \textcolor{gray}{($n*\bar{1}=\bar{n}=\bar{0} \in \rest{n}$)}
\end{bsp}

\begin{defin}
	Seien $A,B \subseteq G, H \leq G$.
	\begin{itemize}
		\item $AB := A*B := \{ a*b : a \in A, b \in B \}$, das Komplexprodukt von $A$ und $B$
		\item $gH := \{g\}*H = \{ g*h : h \in H\}$, die Linksnebenklasse von $H$ bezüglich $g$ \\
		$Hg := H * \{g\} = \{ h*g : h \in H\}$, die Rechtsnebenklasse von $H$ bezüglich $g$
		\item $G/H := \{ gH : g \in G \} \quad$ \textcolor{gray}{(Menge der Linksnebenklassen)} \\
		$H \backslash G := \{ Hg : g \in G \} \quad$ \textcolor{gray}{(Menge der Rechtsnebenklassen)} 
	\end{itemize}
\end{defin}

\begin{bsp}
	Für $h \in H$ ist $hH = H = Hh$ (vgl. 1.4).
\end{bsp}

\begin{lemma}
	Seien $H \leq G, \, g,g' \in G$.
	\begin{enumerate}
		\item $gH = g'H \equivalent g' = gh \in G$ für ein $h \in H$. \\
		$Hg = Hg' \equivalent g' = hg \in G$ für eine $h \in H$
		\item Es ist $gH = g'H$ oder $gH \cup g'H = \emptyset$ und $Hg = Hg'$ oder $Hg \cup Hg' = \emptyset$.
		\item Durch $gH \mapsto Hg^{-1}$ wird eine wohldefinierte Bijektion $G/H \to H \backslash G$ gegeben.
	\end{enumerate}
\end{lemma}
\begin{proof}
	Seien $H \leq G, \, g,g' \in G$.
	\begin{enumerate}
		\item $(\Rightarrow): gH = g'H \follows g' = g' * 1 \in g'H=gH \follows \exists h \in H: g' = gh$. \\
		$(\Leftarrow): g' = gh \follows g'H = ghH \overset{2.7}{=} gH$
		\item Ist $gH \cap g'H \neq \emptyset$, so existieren $h,h' \in H$ mit $gh =g'h'$. $\follows gH=ghH=g'h'H=g'H$
		\item Wohldefiniertheit: $gH=g'H \enspace \overset{\text{a)}}{\Rightarrow} \enspace g'h=gh$ mit $h \in H \enspace \Rightarrow \enspace H(g')^{-1} =Hh^{-1}g^{-1} = Hg^{-1}$ \\
		Bijektivität: klar, da Umkehrabbildung $Hg \mapsto g^{-1}H$
	\end{enumerate}
\end{proof}

\begin{bsppure}
	Betrachte $S_3$ als kleinste nicht-abelsche Gruppe.
\end{bsppure}

\begin{defin}
	Für $H \leq G$ ist
	\begin{align*}
		(G:H) := \# G/H \overset{2.8c}{=} \# h \backslash G \in \nat \cup {\infty}
	\end{align*}
	der Index von $H$ in $G$.
\end{defin}

\begin{bsp}
	$(S_n : A_n) = 2 \qquad (n \geq 2)$ \\
	$(\integer : n\integer) = n$
\end{bsp}

\begin{satz}
	Der Index ist multiplikativ: Sind $K \leq H \leq G$, so ist 
	\begin{align*}
		(G:K) = (G:H) * (H:K)
	\end{align*}
\end{satz}
\begin{proof}
	Nach 2.8b bilden die Nebenklassen von $H$ eine Partition von $G$, d.h. es gibt eine Familie $(g_i)_{i \in I}$ in $G$ mit $G = \bigcup \limits_{i \in I}{g_i H}$ ($G= \bigcup \limits_{i \in I}{g_i H}$ und $g_i H, i \in I$ sind paarweise disjunkt). \\
	Analog ist $H = \bigcup \limits_{j \in J}{h_j K}$ mit $h_j \in H$. Dann gilt: 
	\begin{align*}
		H = \bigcup\limits_{j \in J}{h_j K} &\overset{1.4}{\follows} gH = \bigcup\limits_{j \in J}{g \, h_j \, K} \quad \text{ für jedes } g \in G \\
		&\follows G = \bigcup \limits_{i \in I}{g_i H} = \bigcup \limits_{i \in I}{\bigcup\limits_{j \in J}{g_i h_j K}} = \bigcup \limits_{(i,j) \in I \times J}{g_i h_j K}
	\end{align*}
	Somit ist $(G:K) = \card{I \times J} = \card{I} * \card{J} = (G:H) * (H:K)$.
\end{proof}

\begin{kor} [Satz von Lagrange (wichtigster Satz der Vorlesung)]
	Ist $G$ endlich und $H \leq G$, so ist
	\begin{align*}
		\# G = \# H * (G:H)
	\end{align*}
	Insbesondere gilt $\# H \teilt \# G$ und $(G:H) \teilt \# G$.
\end{kor}
\begin{proof}
	$\# G = (G:1) \overset{2.11}{=} (G:H) * (H:1) = (G:H) * \# H$
\end{proof}

\begin{kor}[Kleiner Satz von Fermat]
	Ist $G$ endlich und $n = \# H$, so ist $g^n=1$ für jedes $g \in G$.
\end{kor}
\begin{proof}
	Nach 2.12 gilt $\ord(g) = \# \erz{g} \teilt \# G = n$. Nach 2.4 ist $g^{\ord(g)}=1$, somit auch
	\begin{align*}
		g^n=(\underbrace{g^{\ord(g)}}_{=1})^{\frac{n}{\ord(g)}} =1
	\end{align*}
\end{proof}
\begin{bem}
	Nach 2.12 ist die Ordnung jeder Untergruppe von $G$ ein Teiler der Gruppenordnung $\# G$. Umgekehrt gibt es im Allgemeinen aber nicht zu jedem Teiler $d$ von $\# G$ eine Untergruppe $H$ von $G$ mit $\# H =d$.
\end{bem}