\section{Die Sylow-Sätze}

Sei $G$ eine endliche Gruppe und $p \in \natural$ prim.
\begin{defin}
	Sei $H \leq G$.
	\begin{enumerate}[label=(\arabic*)]
		\item $H$ ist eine \begriff{$p$-Sylow-Untergruppe} von $G$ (oder kurz $p$-Sylowgruppe) $\quad :\Leftrightarrow \quad$ $H$ ist $p$-Gruppe und $p \teiltnicht (G:H)$
		\item $\Syl_p(G) := \menge{H \leq G: H \text{ ist $p$-Sylowgruppe von } G}$
	\end{enumerate}
\end{defin}

\begin{bem}
	Schreibe $\# G = p^k*m$ mit $p \teiltnicht m$. Dann gilt für $H \leq G$:
	\begin{align*}
		H \in \Syl_p(G) \equivalent \# H = p^k
	\end{align*}
\end{bem}

\begin{bsp}
	$\Syl_3(S_3) = \menge{A_3} \qquad \Syl_2(S_3) = \menge{\erz{(1\,2)}, \erz{(1\,3)}, \erz{(2\,3)}}$ \\
	$D_4 \in \Syl_2(S_4)$
\end{bsp}

\begin{satz}
	$\Syl_p(G) \neq \emptyset$.
\end{satz}
\begin{proof}
	Induktion nach $n = \# G = p^k * m, \enspace p \teiltnicht m$.
	Für $n=1$ ist $1 \in \Syl_p(1)$. \\
	Sei nun $n>1$. Ist $p \teiltnicht n$, so ist $1 \in \Syl_p(G)$. Sei also $k \geq 1$.
	\begin{itemize}
		\item 1.Fall: Es ex. ein $H \lneqq G$ mit $p \teiltnicht (G:H)$. Nach Induktionshypothese existiert $S \in \Syl_p(H)$. Da $p \teiltnicht (G:S) = (G:H)(H:S)$ ist $S \in \Syl_p(G)$.
		\item Fall 2: Es ist $p \teilt (G:H)$ für alle $H \leq G$. Nach Klassengleichung 6.16 ist $0 = n  = \# Z(G) + \sum_{i+1}^{r}{(G:C_i(x_i))} = \# Z(G) \mod p$, wobei $G/Z(G)=\biguplus_{i=1}^{r}{x_i^G}$, also $p \teilt \# Z(G)$. Nach 7.3 existiert $g \in Z(G)$ mit $\ord(g) = p$. \\
		$\follows N := \erz{g} \normalteiler G$, $\# N = p$, $\# G/N = p^{k-1}*m$ \\
		Nach Induktionshypothese existiert also $\quer{S} \in \Syl_p(G/N)$, d.h. $\# \quer{S} = p^{k-1}$. Setze $S := \pi_N^{-1}(\quer{S}) \leq G$. Dann ist $\# S = \# \Ker(\pi_N) * \# \quer{S} = p*p^{k-1} = p^k$, d.h. $S \in \Syl_p(G)$.
	\end{itemize}
\end{proof}

\begin{kor}
	Ist $k \in \nat$ mit $p^k \teilt \# G$, so existiert $H \leq G$  mit $\# H = p^k$.
\end{kor}
\begin{proof}
	Anwendung von 8.4 und 7.9
\end{proof}

\begin{thm}[Sylow-Sätze]
	Sei $G$ eine endliche Gruppe.
	\begin{enumerate}[label=(\roman*)]
		\item Jede $p$-Gruppe $H \leq G$ ist in einer $p$-Sylowgruppe von $G$ enthalten.
		\item Je zwei $p$-Sylowgruppen von $G$ sind konjugiert.
		\item Für die Anzahl $s_p := \# \Syl_p(G)$ gilt
		\begin{align*}
			s_p = (G:N_G(S)) \equiv 1 \mod p
		\end{align*}
		wobei $S \in \Syl_p(G)$ beliebig.
	\end{enumerate}
\end{thm}
\begin{proof}
	Fixiere $S_0 \in \Syl_p(G)$ (existiert nach 8.4). Definiere $X := \menge{S_0^g : g \in G} \subseteq \Syl_p(G)$.
\end{proof}