\newcommand{\begriff}[1]{\textit{#1}}

% |--------------------------|
% | Theoreme                 |
% |--------------------------|

\newcommand{\boxskip}{7pt}        % Abstand in Boxen zwischen Rand und Text
\newcommand{\skiparound}{10pt}    % Abstand vor und nach Theoremen
\newcommand{\thmstyle}{changebreak}     % Sytle für Theorem-Umgebungen

\theoremstyle{\thmstyle}
\theorembodyfont{}                %setzt bodyfont auf nicht kursiv

\theorempreskip{\skiparound}      %Abstände
\theorempostskip{\skiparound}


%---------Theorem-------------
\newmdtheoremenv[%
	backgroundcolor=lightgrey,%
	%linecolor=darkgrey,%
	%innertopmargin=\boxskip,%
	%innerbottommargin=\boxskip,%
	%topline=true,%
	%rightline=true,%
	%leftline=true,%
	%bottomline=true,%
	%innertopmargin=3pt,%
	%innerbottommargin=3pt,%
	leftmargin=-10pt,%
	rightmargin=-10pt,%
	%frametitlefont=\normalfont\bfseries\color{black},%
	%skipabove=5pt,%
	%skipbelow=5pt,%
	skipabove=\skiparound,%
	skipbelow=\skiparound,%
]{thm}[themcount]{Theorem}

%---------Satz----------------
\newmdtheoremenv[%
	%backgroundcolor=lightgrey,%
	%linecolor=darkgrey,%
	%innertopmargin=\boxskip,%
	%innerbottommargin=\boxskip,%
	%linecolor=darkgrey,%
	%topline=true,%
	%rightline=true,%
	%leftline=true,%
	%bottomline=true,%
	%backgroundcolor=lightgrey,%
	%innertopmargin=3pt,%
	%innerbottommargin=3pt,%
	leftmargin=-10pt,%
	rightmargin=-10pt,%
	%frametitlefont=\normalfont\bfseries\color{black},%
	%skipabove=5pt,%
	%skipbelow=5pt,%
	skipabove=\skiparound,%
	skipbelow=\skiparound,%
]{satz}[themcount]{Satz}
%\newtheorem{satz}[themcount]{Satz}

%---------Definition-----------
%\theoremstyle{nonumberbreak}
%\newmdtheoremenv[%
%	outerlinewidth=3pt,%
%	linecolor=darkgrey,%
%	topline=false,%
%	rightline=false,%
%	bottomline=false,%
%	innertopmargin=3pt,%
%	innerbottommargin=3pt,%
%	frametitlefont=\normalfont\bfseries\color{black},%
%	skipabove=5pt,%
%	skipbelow=5pt,%
%]{defin}[themcount]{Definition}
%\theoremstyle{\thmstyle}
\newtheorem{defin}[themcount]{Definition}


%---------Lemma---------------
\newtheorem{lemma}[themcount]{Lemma}

%---------Folgerung--------------
\newtheorem{folg}[themcount]{Folgerung}

%---------Korollar--------------
\newtheorem{kor}[themcount]{Korollar}

%---------Beispiel----------------
\newtheorem{bsp}[themcount]{Beispiel}

%---------Erinnerung--------------
\newtheorem{erinnerung}[themcount]{Erinnerung}

%---------Bemerkung--------------
%\theoremstyle{nonumberplain}
\newtheorem{bem}[themcount]{Bemerkung}

\theoremstyle{nonumberplain}
\theorembodyfont{\normalfont}
\newtheorem{bsppure}{Beispiel}


%---------Übungsblatt--------------


%---------Hausaufgaben--------------
%\newenvironment{hausaufgaben}[args]{%
%	\stepcounter{blattcount}%
%	\fcolorbox{black}{lightgrey}{%
%		\begin{minipage}{0.5\textwidth}
%			\textsf{{\huge \textbf{Hausaufgaben}}} \\
%			\textsf{Übungsblatt \theblattcount}
%		\end{minipage}
%		\begin{minipage}{0.5\textwidth}
%			\flushright \textbf{Eric Kunze} (Matr.-Nr. 4679202)\\
%			Ü-Gruppe: gerade Woche (M. Schönherr)
%		\end{minipage}%
%	}{%
%	enddef
%}

%\newcommand{\aufgabe}[1]{{\Large \textsf{\textbf{Aufgabe #1}}}}
%\newcommand{\teilaufgabe}[1]{\textsf{\textit{{\large Teilaufgabe (#1)}}}}
%
%\newcommand{\headeraufgabenblatt}{%
%	\stepcounter{blattcount}%
%	\fcolorbox{black}{lightgrey}{%
%		\begin{minipage}{0.5\textwidth}
%			\textsf{{\huge \textbf{Hausaufgaben}}} \\
%			\textsf{Übungsblatt \theblattcount}
%		\end{minipage}
%		\begin{minipage}{0.5\textwidth}
%			\flushright \textbf{Eric Kunze} (Matr.-Nr. 4679202)\\
%			Ü-Gruppe: gerade Woche (M. Schönherr)
%		\end{minipage}%
%	} \setcounter{figure}{0} \setcounter{equation}{0} \\[5mm]}

