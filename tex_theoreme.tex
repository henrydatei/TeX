%
% ================================ 
% | THEOREME & UMGEBUNGEN        |
% ================================

% >> Counter für Umgebungen
\newcounter{themcount}
\newcounter{defcount}
\newcounter{taskcount}
\newcounter{blattcount}
%\renewcommand{\thethemcount}{\relax}

% >> Eigenschaften für Übungsblätter
\newcommand{\uebungsgruppe}{Tag x. DS, (un)gerade Woche}
\newcommand{\name}{Eric Kunze}
\newcommand{\matrikelnr}{4679202}

% >> Überschriften
\newcommand{\aufgabe}[1]{{\Large \textsf{\textbf{Aufgabe #1}}}}
\newcommand{\teilaufgabe}[1]{\textsf{\textit{{\large Teilaufgabe (#1)}}}}

% >> Farben
\definecolor{lightgrey}{gray}{0.89}
\definecolor{darkgrey}{gray}{0.6}

\definecolor{blue}{rgb}{50,85,150}
\definecolor{lightred}{rgb}{1,0.6,0.6}
\definecolor{darkgreen}{rgb}{0,0.6,0}

% >> Theoreme

\newcommand{\boxskip}{7pt}        		% Abstand in Boxen zwischen Rand und Text
\newcommand{\skiparound}{10pt}   		% Abstand vor und nach Theoremen
\newcommand{\thmstyle}{changebreak}     % Sytle für Theorem-Umgebungen

\theoremstyle{\thmstyle}
\theorembodyfont{}                % setzt bodyfont auf nicht kursiv

\theorempreskip{\skiparound}      % Abstände
\theorempostskip{\skiparound}


%---------Theorem-------------
\newmdtheoremenv[%
	backgroundcolor=lightgrey,%
	%linecolor=darkgrey,%
	%innertopmargin=\boxskip,%
	%innerbottommargin=\boxskip,%
	%topline=true,%
	%rightline=true,%
	%leftline=true,%
	%bottomline=true,%
	%innertopmargin=3pt,%
	%innerbottommargin=3pt,%
	leftmargin=-10pt,%
	rightmargin=-10pt,%
	%frametitlefont=\normalfont\bfseries\color{black},%
	%skipabove=5pt,%
	%skipbelow=5pt,%
	skipabove=\skiparound,%
	skipbelow=\skiparound,%
]{thm}[themcount]{Theorem}

%---------Satz----------------
\newmdtheoremenv[%
	%backgroundcolor=lightgrey,%
	%linecolor=darkgrey,%
	%innertopmargin=\boxskip,%
	%innerbottommargin=\boxskip,%
	%linecolor=darkgrey,%
	%topline=true,%
	%rightline=true,%
	%leftline=true,%
	%bottomline=true,%
	%backgroundcolor=lightgrey,%
	%innertopmargin=3pt,%
	%innerbottommargin=3pt,%
	leftmargin=-10pt,%
	rightmargin=-10pt,%
	%frametitlefont=\normalfont\bfseries\color{black},%
	%skipabove=5pt,%
	%skipbelow=5pt,%
	skipabove=\skiparound,%
	skipbelow=\skiparound,%
]{satz}[themcount]{Satz}
%\newtheorem{satz}[themcount]{Satz}

%---------Definition-----------
%\theoremstyle{nonumberbreak}
%\newmdtheoremenv[%
%	outerlinewidth=3pt,%
%	linecolor=darkgrey,%
%	topline=false,%
%	rightline=false,%
%	bottomline=false,%
%	innertopmargin=3pt,%
%	innerbottommargin=3pt,%
%	frametitlefont=\normalfont\bfseries\color{black},%
%	skipabove=5pt,%
%	skipbelow=5pt,%
%]{defin}[themcount]{Definition}
%\theoremstyle{\thmstyle}
\newtheorem{defin}[themcount]{Definition}


%---------Lemma---------------
\newtheorem{lemma}[themcount]{Lemma}

%---------Folgerung--------------
\newtheorem{folg}[themcount]{Folgerung}

%---------Korollar--------------
\newtheorem{kor}[themcount]{Korollar}

%---------Beispiel----------------
\newtheorem{bsp}[themcount]{Beispiel}

%---------Erinnerung--------------
\newtheorem{erinnerung}[themcount]{Erinnerung}

%---------Bemerkung--------------
%\theoremstyle{nonumberplain}
\newtheorem{bem}[themcount]{Bemerkung}

\theoremstyle{nonumberplain}
\theorembodyfont{\normalfont}
\newtheorem{bsppure}{Beispiel}

%--------------------------------
\theoremstyle{break}

\theoremheaderfont{\sffamily\bfseries\large}
%
%--------------------------------
%
%----------Übung-----------------
\newmdtheoremenv[%
%backgroundcolor=lightgrey,%
%linecolor=darkgrey,%
%innertopmargin=\boxskip,%
%innerbottommargin=\boxskip,%
%topline=true,%
%rightline=true,%
%leftline=true,%
%bottomline=true,%
%innertopmargin=3pt,%
%innerbottommargin=3pt,%
%leftmargin=-10pt,%
%rightmargin=-10pt,%
%frametitlefont=\sffamily\bfseries\color{black},%
%skipabove=5pt,%
%skipbelow=5pt,%
skipabove=\skiparound,%
skipbelow=\skiparound,%
]{uebung}[taskcount]{Übung}
%
%-----------Lösung-----------------
\newenvironment{loesung}{%
	\vspace{-2mm}
	\textit{Lösung:} \\%
}{}



%---------Übungsblatt--------------


%---------Hausaufgaben--------------
\newenvironment{hausaufgabe}{%
	\stepcounter{blattcount}%
	\fcolorbox{black}{lightgrey}{%
		\begin{minipage}{0.5\textwidth}
			\textsf{{\huge \textbf{Hausaufgaben}}} \\
			\textsf{Übungsblatt \theblattcount}
		\end{minipage}
		\begin{minipage}{0.5\textwidth}
			\flushright \textbf{\name} (Matr.-Nr. \matrikelnr)\\
			Ü-Gruppe: \uebungsgruppe
		\end{minipage}}%
	\setcounter{figure}{0}%
	\setcounter{equation}{0}%
	\setcounter{table}{0}%
	\vspace{5mm}%
	}{%
	\newpage%
}

%
%
%\newcommand{\headeraufgabenblatt}{%
%	\stepcounter{blattcount}%
%	\fcolorbox{black}{lightgrey}{%
%		\begin{minipage}{0.5\textwidth}
%			\textsf{{\huge \textbf{Hausaufgaben}}} \\
%			\textsf{Übungsblatt \theblattcount}
%		\end{minipage}
%		\begin{minipage}{0.5\textwidth}
%			\flushright \textbf{Eric Kunze} (Matr.-Nr. 4679202)\\
%			Ü-Gruppe: gerade Woche (M. Schönherr)
%		\end{minipage}%
%	} \setcounter{figure}{0} \setcounter{equation}{0} \\[5mm]}

